\documentclass[10pt]{article}
\usepackage[T1]{fontenc}
\usepackage[utf8]{inputenc}%ATTENTION codage UTF8
\usepackage{fourier}
\usepackage[scaled=0.875]{helvet}
\renewcommand{\ttdefault}{lmtt}
\usepackage{amsmath,amssymb,makeidx}
\usepackage[normalem]{ulem}
\usepackage{mathrsfs}
\usepackage{diagbox}
\usepackage{fancybox}
\usepackage{tabularx,booktabs}
\usepackage{colortbl}
\usepackage{pifont}
\usepackage{multirow}
\usepackage{dcolumn}
\usepackage{enumitem}
\usepackage{textcomp}
\usepackage{lscape}
\usepackage{cancel}
\newcommand{\textding}[1]{\text{\ding{#1}}}
\newcommand{\euro}{\eurologo{}}
\usepackage{graphics,graphicx}
\usepackage{pstricks,pst-plot,pst-tree,pstricks-add}
\usepackage{pst-eucl}  % permet de faire des dessins de géométrie simplement
\usepackage{pst-text}
\usepackage{pst-node,pst-all}
\usepackage[left=3.5cm, right=3.5cm, top=3cm, bottom=3cm]{geometry}
\renewcommand{\pstEllipse}[5][]{% arc d'ellipse pour le sujet de Polynésie septembre 2013
\psset{#1}
\parametricplot{#4}{#5}{#2\space t cos mul #3\space t sin mul}}
\newcommand{\R}{\mathbb{R}}
\newcommand{\N}{\mathbb{N}}
\newcommand{\D}{\mathbb{D}}
\newcommand{\Z}{\mathbb{Z}}
\newcommand{\Q}{\mathbb{Q}}
\newcommand{\C}{\mathbb{C}}
\usepackage{scratch}
\renewcommand{\theenumi}{\textbf{\arabic{enumi}}}
\renewcommand{\labelenumi}{\textbf{\theenumi.}}
\renewcommand{\theenumii}{\textbf{\alph{enumii}}}
\renewcommand{\labelenumii}{\textbf{\theenumii.}}
\newcommand{\vect}[1]{\overrightarrow{\,\mathstrut#1\,}}
\def\Oij{$\left(\text{O}~;~\vect{\imath},~\vect{\jmath}\right)$}
\def\Oijk{$\left(\text{O}~;~\vect{\imath},~\vect{\jmath},~\vect{k}\right)$}
\def\Ouv{$\left(\text{O}~;~\vect{u},~\vect{v}\right)$}
\usepackage{fancyhdr}
\usepackage{marvosym}   %c'est pour le symbole euro : code \EUR{}
\usepackage[french]{babel}
\usepackage[dvips]{hyperref}
\usepackage[np]{numprint}
%Tapuscrit : Denis Vergès
%\frenchbsetup{StandardLists=true}

%\usepackage{tkz-tab,tkz-fct}%,tkz-tukey} % à commenter pour amérique nord 2021
\usepackage{xkeyval,ifthen}
\usepackage{multicol,xcolor,framed}
%\usepackage{tkz-euclide} % à commenter pour amérique nord 2021
%\usetkzobj{all}
\usepackage[tikz]{bclogo}
%\usetikzlibrary{babel,arrows,shadows,decorations.pathmorphing,decorations.markings,patterns} % à commenter pour amérique nord 2021
\usetikzlibrary{backgrounds}
\usetikzlibrary{calc,shapes.arrows,decorations.pathreplacing}
\usepackage{pgf,pgfplots}

\usepackage{tikz} % arbre en proba
\usetikzlibrary{trees} % arbre en proba
\usepackage{forest} % arbre en proba
\usetikzlibrary{positioning}
 % Structure servant à avoir l'événement et la probabilité.
\def\getEvene#1/#2\endget{$#1$}
\def\getProba#1/#2\endget{$#2$}
\input{xlop} % JM pour les opérations
%%% Table des nombres premiers  %%%%
\newcount\primeindex
\newcount\tryindex
\newif\ifprime
\newif\ifagain
\newcommand\getprime[1]{%
\opcopy{2}{P0}%
\opcopy{3}{P1}%
\opcopy{5}{try}
\primeindex=2
\loop
\ifnum\primeindex<#1\relax
\testprimality
\ifprime
\opcopy{try}{P\the\primeindex}%
\advance\primeindex by1
\fi
\opadd*{try}{2}{try}%
\ifnum\primeindex<#1\relax
\testprimality
\ifprime
\opcopy{try}{P\the\primeindex}%
\advance\primeindex by1
\fi
\opadd*{try}{4}{try}%
\fi
\repeat
}
\newcommand\testprimality{%
\begingroup
\againtrue
\global\primetrue
\tryindex=0
\loop
\opidiv*{try}{P\the\tryindex}{q}{r}%
\opcmp{r}{0}%
\ifopeq \global\primefalse \againfalse \fi
\opcmp{q}{P\the\tryindex}%
\ifoplt \againfalse \fi
\advance\tryindex by1
\ifagain
\repeat
\endgroup
}
%%% Décomposition en nombres premiers %%%

\newcommand\primedecomp[2][nil]{%
\begingroup
\opset{#1}%
\opcopy{#2}{NbtoDecompose}%
\opabs{NbtoDecompose}{NbtoDecompose}%
\opinteger{NbtoDecompose}{NbtoDecompose}%
\opcmp{NbtoDecompose}{0}%
\ifopeq
Je refuse de décomposer zéro.
\else
\setbox1=\hbox{\opdisplay{operandstyle.1}%
{NbtoDecompose}}%
{\setbox2=\box2{}}%
\count255=1
\primeindex=0
\loop
\opcmp{NbtoDecompose}{1}\ifopneq
\opidiv*{NbtoDecompose}{P\the\primeindex}{q}{r}%
\opcmp{0}{r}\ifopeq
\ifvoid2
\setbox2=\hbox{%
\opdisplay{intermediarystyle.\the\count255}%
{P\the\primeindex}}%
\else
\setbox2=\vtop{%
\hbox{\box2}
\hbox{%
\opdisplay{intermediarystyle.\the\count255}%
{P\the\primeindex}}}
\fi
\opcopy{q}{NbtoDecompose}%
\advance\count255 by1
\setbox1=\vtop{%
\hbox{\box1}
\hbox{%
\opdisplay{operandstyle.\the\count255}%
{NbtoDecompose}}
}%
\else
\advance\primeindex by1
\fi
\repeat
\hbox{\box1
\kern0.5\opcolumnwidth
\opvline(0,0.75){\the\count255.25}
\kern0.5\opcolumnwidth
\box2}%
\fi
\endgroup
}