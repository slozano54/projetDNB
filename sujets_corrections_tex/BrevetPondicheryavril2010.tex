\documentclass[10pt]{article}
\usepackage[T1]{fontenc}
\usepackage[utf8]{inputenc}
\usepackage{fourier}
\usepackage[scaled=0.875]{helvet} 
\renewcommand{\ttdefault}{lmtt} 
\usepackage{amsmath,amssymb,makeidx}
\usepackage[normalem]{ulem}
\usepackage{fancybox}
\usepackage{tabularx}
\usepackage{ulem}
\usepackage{dcolumn}
\usepackage{textcomp}
\usepackage{tabularx}
\usepackage{lscape}
\newcommand{\euro}{\eurologo{}}
%Tapuscrit : Denis Vergès
\usepackage{pstricks,pst-plot,pst-text,pst-tree}
\setlength\paperheight{297mm}
\setlength\paperwidth{210mm}
\setlength{\textheight}{25cm}
\newcommand{\R}{\mathbb{R}}
\newcommand{\N}{\mathbb{N}}
\newcommand{\D}{\mathbb{D}}
\newcommand{\Z}{\mathbb{Z}}
\newcommand{\Q}{\mathbb{Q}}
\newcommand{\C}{\mathbb{C}}
\renewcommand{\theenumi}{\textbf{\arabic{enumi}}}
\renewcommand{\labelenumi}{\textbf{\theenumi.}}
\renewcommand{\theenumii}{\textbf{\alph{enumii}}}
\renewcommand{\labelenumii}{\textbf{\theenumii.}}
\newcommand{\vect}[1]{\mathchoice%
{\overrightarrow{\displaystyle\mathstrut#1\,\,}}%
{\overrightarrow{\textstyle\mathstrut#1\,\,}}%
{\overrightarrow{\scriptstyle\mathstrut#1\,\,}}%
{\overrightarrow{\scriptscriptstyle\mathstrut#1\,\,}}}
\def\Oij{$\left(\text{O},~\vect{\imath},~\vect{\jmath}\right)$}
\def\Oijk{$\left(\text{O},~\vect{\imath},~\vect{\jmath},~\vect{k}\right)$}
\def\Ouv{$\left(\text{O},~\vect{u},~\vect{v}\right)$}
\setlength{\voffset}{-1,5cm}
\usepackage{fancyhdr}
\usepackage[dvips]{hyperref}
\hypersetup{%
pdfauthor = {APMEP},
pdfsubject = {Brevet des collèges},
pdftitle = {Pondichéry avril 2010},
allbordercolors = white}  
\thispagestyle{empty}
\usepackage[frenchb]{babel}
\usepackage[np]{numprint}
\begin{document}
\setlength\parindent{0mm}
\rhead{\textbf{A. P{}. M. E. P{}.}}
\lhead{\small Brevet des collèges}
\lfoot{\small{Pondichéry}}
\cfoot{\small{avril 2010}}
\renewcommand \footrulewidth{.2pt}
\pagestyle{fancy}
\thispagestyle{empty}
\begin{center}{\Large{ \textbf{\decofourleft~Brevet des collèges Pondichéry~\decofourright\\
avril 2010}}} 

\end{center}

\vspace{0,25cm}

\textbf{Activités numériques}

\bigskip

\textbf{\textsc{Exercice 1} }

\medskip

Une classe de 3\up{e} est constituée de 25 élèves.

Certains sont externes, les autres sont demi-pensionnaires.

Le tableau ci-dessous donne la composition de la classe.

\medskip

\[\begin{tabularx}{0.8\linewidth}{|c|*{3}{>{\centering \arraybackslash}X|}}\cline{2-4}
\multicolumn{1}{c|}{}	&Garçon	& Fille	& Total\\ \hline
Externe					& \ldots 	&3		& \ldots\\ \hline
Demi-pensionnaire		& 9		& 11	& ...\\ \hline
Total					& \ldots	& \ldots	& 25\\ \hline
\end{tabularx}\]
 
\medskip
 
\begin{enumerate}
\item  Recopier et compléter le tableau.
\item  On choisit au hasard un élève de cette classe.
	\begin{enumerate}
		\item Quelle est la probabilité pour que cet élève soit une fille ?
		\item Quelle est la probabilité pour que cet élève soit externe ?
		\item Si cet élève est demi-pensionnaire, quelle est la probabilité que ce soit un garçon ?
	\end{enumerate}
\end{enumerate}

\bigskip

\textbf{\textsc{Exercice 2} }

\medskip

On donne :

\[\text{A} = \dfrac{6}{5} - \dfrac{17}{14}\div  \dfrac{5}{7} \qquad  \text{B} = \dfrac{8 \times 10^8 \times 1,6}{0,4 \times 10^{-3}}\qquad 
\text{C} = \left(\sqrt{5} + \sqrt{10}\right)^2 - 10\sqrt{2}.\]
  
\begin{enumerate}
\item  Écrire A sous la forme d'une fraction irréductible.
\item  Donner l'écriture scientifique de B.
\item  Montrer que C est un nombre entier.
\end{enumerate}

\bigskip

\textbf{\textsc{Exercice 3} }

\medskip
Pour chaque question, écrire la lettre correspondant à la bonne réponse.
Aucune justification n'est demandée.

\medskip

\begin{tabularx}{\linewidth}{|c|p{3.75cm}|*{3}{>{\centering \arraybackslash}X|}}\cline{3-5}
\multicolumn{2}{c|}{}&\multicolumn{3}{c|}{Réponses}\\ \hline
	&Questions	&A	&B	&C\\ \hline
1&\begin{tabular}{p{3.65cm}}
Quelle expression est égale à 6 si on choisit la valeur $x =-1$ ?\\
\end{tabular} &
$-3x^2$&$6(x + 1)$&$5x^2+ 1$\\ \hline
2&\begin{tabular}{p{3.65cm}}Le développement de \mbox{$(x + 3)(2x + 4) -2(5x + 6)$} est :\\
\end{tabular}&
$2x^2$	&$2x^2 + 20x + 24$	& $2x^2+ 24$\\ \hline
3&
\begin{tabular}{p{3.65cm}}La factorisation de \mbox{$9x^2-16$} est:\\
\end{tabular}&$(3x - 4)^2$&$(3x + 4)(3x - 4)$&$(3x + 4)^2$
	\\ \hline
4&\begin{tabular}{p{3.65cm}}Les solutions de l'équation $(x - 5)(3x + 4) = 0$ sont:\\
\end{tabular}&
$\dfrac{4}{3}$ et 5&$- \dfrac{4}{3}$ et 5&$\dfrac{4}{3}$ et $-5$ \\ \hline
\end{tabularx}

\newpage

\textbf{Activités géométriques}

\bigskip

\textbf{\textsc{Exercice 1} }

\medskip

\begin{enumerate}
\item  Construire un triangle ABC tel que : AB = 7,5~cm ; BC = 10~cm et AC = 12,5~cm.
\item  Prouver que le triangle ABC est rectangle en B.
\item  
	\begin{enumerate}
		\item  Construire le point F appartenant au segment [AC] tel que CF = 5~cm.
		\item Construire le point G appartenant au segment [BC] tel que CG = 4~cm.
	\end{enumerate}
\item  Montrer que les droites (AB) et (FG) sont parallèles.
\item  Montrer que la longueur FG est égale à 3~cm.
\item  Les droites (FG) et (BC) sont-elles perpendiculaires ? Justifier.
\end{enumerate}

\bigskip

\textbf{\textsc{Exercice 2} }

\medskip
En travaux pratiques de chimie, les élèves utilisent des récipients, appelés erlenmeyers, comme celui schématisé ci -dessous.

\medskip

\psset{unit=1cm,arrowsize=3pt 3}
\begin{center}
\begin{pspicture}(-2.5,0)(2.5,6)
\pscurve[linewidth=1.25pt](-1.95,0.9)(-2.,0.8)(-1.5,0.3)(-1,0.12)(0,0)(1,0.12)(1.5,0.3)(2.,0.8)(1.95,0.9)
\psline(-2,0.85)(-0.8,3.8)(-0.8,5.7)
\psline(2,0.85)(0.8,3.8)(0.8,5.7)
\psellipse[linestyle=dashed](0,0.8)(2,0.8)
\psline[linestyle=dashed](-0.8,3.8)(0,5.7)(0.8,3.8)
\psellipse(0,5.7)(0.8,0.3)
\psellipse[linestyle=dashed](0,3.78)(0.8,0.3)
\pscurve[linewidth=1.pt](-0.8,3.8)(-0.7,3.66)(-0.6,3.58)(-0.4,3.53)(0,3.5)(0.4,3.53)(0.6,3.58)(0.7,3.66)(0.8,3.8)
\uput[l](-2,3.8){Niveau maximum de l'eau}
\psline[linestyle=dashed](0,5.7)(0,0.8)(2,0.8)
\psline[linestyle=dashed](-0.8,3.8)(0.8,3.8)
\psline(0,3.95)(0.15,3.95)(0.15,3.8)
\psline(0,0.95)(0.15,0.95)(0.15,0.8)
\uput[ul](0,5.6){S} \uput[ul](0.05,3.8){O$'$} \uput[ur](0.8,3.8){B$'$}
\uput[l](0,0.8){O} \uput[r](2,0.8){B}
\psline{->}(-2,3.8)(-1,3.8)
\end{pspicture}
\end{center}


Le récipient est rempli d'eau jusqu'au niveau maximum indiqué sur le schéma par une flèche.

On note :

\quad  C$_{1}$ le grand cône de sommet S et de base le disque de centre O et de rayon OB.

\quad  C$_{2}$ le petit cône de sommet S et de base le disque de centre O$'$ et de rayon O$'$B$'$.

On donne : SO = 12~cm et OB = 4~cm

\medskip

\begin{enumerate}
\item  Le volume $V$ d'un cône de révolution de rayon $R$ et de hauteur $h$ est donné par la formule :

\[ V = \dfrac{1}{3}\times \pi \times R^2 \times h\]

Calculer la valeur exacte du volume du cône C$_{1}$.
\item  Le cône C$_{2}$ est une réduction du cône C$_{1}$. On donne SO$'$ = 3~cm.
	\begin{enumerate}
		\item  Quel est le coefficient de cette réduction ?
		\item  Prouver que la valeur exacte du volume du cône C$_{2}$ est égale à $\pi$~cm$^3$.
	\end{enumerate}
\item 
	\begin{enumerate}
		\item  En déduire que la valeur exacte du volume d'eau contenue dans le récipient, en cm$^3$, est $63\pi$. 
		\item  Donner la valeur approchée de ce volume d'eau arrondie au cm$^3$ près. 
	\end{enumerate}
\item Ce volume d'eau est-il supérieur à 0,2~litres ? Expliquer pourquoi.
\end{enumerate}

\newpage

\textbf{Problème \hfill 12 points}

\medskip

\begin{center}
\textbf{Les trois parties sont indépendantes}
\end{center}

\textbf{Partie 1}

\medskip

Un disquaire en ligne propose de télécharger légalement de la musique.

\medskip

\setlength\parindent{5mm}
\begin{itemize}
\item Offre A : 1,20~\euro{} par morceau téléchargé avec un accès gratuit au site.
\item Offre B : 0,50~\euro{} par morceau téléchargé moyennant un abonnement annuel de 35~\euro.
\end{itemize}
\setlength\parindent{0mm}

\medskip

\begin{enumerate}
\item  Calculer, pour chaque offre, le prix pour 30 morceaux téléchargés par an.
\item  
	\begin{enumerate}
		\item  Exprimer, en fonction du nombre $x$ de morceaux téléchargés, le prix avec l'offre A. 
		\item  Exprimer, en fonction du nombre $x$ de morceaux téléchargés, le prix avec l'offre B.
	\end{enumerate}
\item  Soit $f$ et $g$ les deux fonctions définies par : 

\[f : \quad x \longmapsto 1,2x\quad  \text{et} \quad  g : \quad  x \longmapsto 0,5x+35.\]

	\begin{enumerate}
		\item  L'affirmation ci-dessous est-elle correcte? Expliquer pourquoi.
		
\og $f$ et $g$ sont toutes les deux des fonctions linéaires \fg.
		\item  Représenter sur la feuille de papier millimétré, dans un repère orthogonal les représentations graphiques des fonctions $f$ et $g$. On prendra 1 cm pour 10~morceaux en abscisse et 1~cm pour 10~\euro{} en ordonnée.
 	\end{enumerate}
\item  Déterminer le nombre de morceaux pour lequel les prix sont les mêmes.
\item  Déterminer l'offre la plus avantageuse si on achète 60 morceaux à l'année.
\item  Si on dépense 80~\euro, combien de morceaux peut-on télécharger avec l'offre B ?
\end{enumerate}

\medskip

\textbf{Partie 2}

\medskip

On admet qu'un morceau de musique représente 3 Mo de mémoire. (1 Mo = 1 méga-octet)

\medskip

\begin{enumerate}
\item  Combien de morceaux de musique peut-on télécharger sur une clé USB d'une capacité de stockage de 256 Mo ?

La vitesse de téléchargement d'un morceau de musique sur le site est de 10 Mo/s. (méga-octet par seconde)
\item  Combien de morceaux peut-on télécharger en deux minutes ?
\end{enumerate}
 
\medskip

\textbf{Partie 3}

\medskip

Les créateurs du site réalisent une enquête de satisfaction auprès des internautes clients.

Ils leur demandent d'attribuer une note sur 20 au site.

Le tableau suivant donne les notes de 50 internautes.

\medskip

\begin{tabularx}{\linewidth}{|c|*{7}{>{\centering \arraybackslash}X|}}\hline
Note	&6	&8	&10	&12	&14	&15	&17\\ \hline
Effectif&1	&5	&7	&8	&12	&9	&8\\ \hline
\end{tabularx}

\medskip

\begin{enumerate}
\item  Calculer la note moyenne obtenue par le site. Arrondir le résultat à l'unité.
\item  L'enquête est jugée satisfaisante si 55\:\% des internautes ont donné une note supérieure ou égale à 14. Est-ce le cas? Expliquer pourquoi.
\end{enumerate}
\end{document}