\documentclass[10pt]{article}
\usepackage[T1]{fontenc}
\usepackage[utf8]{inputenc}
\usepackage{fourier}
\usepackage[scaled=0.875]{helvet}
\renewcommand{\ttdefault}{lmtt}
\usepackage{amsmath,amssymb,makeidx}
\usepackage[normalem]{ulem}
\usepackage{fancybox}
\usepackage{tabularx}
\usepackage{diagbox}
\usepackage{ulem}
\usepackage{pifont}
\usepackage{dcolumn}
\usepackage{scratch}
\usepackage{enumitem}
\usepackage{multirow}
\usepackage{textcomp}
\usepackage{multicol}
\usepackage{lscape}
%Tapuscrit : Denis Vergès
\newcommand{\euro}{\eurologo{}}
\usepackage{graphicx}
\usepackage{pstricks,pst-plot,pst-tree,pst-eucl,pstricks-add}
\usepackage[left=3.5cm, right=3.5cm, top=3cm, bottom=3cm]{geometry}
\newcommand{\R}{\mathbb{R}}
\newcommand{\N}{\mathbb{N}}
\newcommand{\D}{\mathbb{D}}
\newcommand{\Z}{\mathbb{Z}}
\newcommand{\Q}{\mathbb{Q}}
\newcommand{\C}{\mathbb{C}}
\renewcommand{\sfdefault}{phv}% police helvetica pour les blocs scratch.
%\usepackage{scratch3}
\renewcommand{\theenumi}{\textbf{\arabic{enumi}}}
\renewcommand{\labelenumi}{\textbf{\theenumi.}}
\renewcommand{\theenumii}{\textbf{\alph{enumii}}}
\renewcommand{\labelenumii}{\textbf{\theenumii.}}
\newcommand{\vect}[1]{\overrightarrow{\,\mathstrut#1\,}}
\def\Oij{$\left(\text{O}~;~\vect{\imath},~\vect{\jmath}\right)$}
\def\Oijk{$\left(\text{O}~;~\vect{\imath},~\vect{\jmath},~\vect{k}\right)$}
\def\Ouv{$\left(\text{O}~;~\vect{u},~\vect{v}\right)$}
\usepackage{fancyhdr}
\usepackage[french]{babel}
\usepackage[dvips]{hyperref}
\hypersetup{%
pdfauthor = {APMEP},
pdfsubject = {Brevet des collèges},
pdftitle = { 14 septembre 2020},
allbordercolors = white,
pdfstartview=FitH}  
\usepackage[np]{numprint}
\frenchbsetup{StandardLists=true}
\begin{document}
\setlength\parindent{0mm}
\rhead{\textbf{A. P{}. M. E. P{}.}}
\lhead{\small Brevet des collèges}
\lfoot{\small{ Antilles Guyane}}
\rfoot{\small{14 septembre  2020}}
\pagestyle{fancy}
\thispagestyle{empty}
\begin{center}
    
{\Large \textbf{\decofourleft~Brevet des collèges Antilles--Guyane~\decofourright}\\[5pt]\textbf{14 septembre  2020}}
    
\bigskip
    
\textbf{Durée : 2 heures} 

\medskip

\end{center}

\textbf{Exercice 1 \hfill 20 points}

\medskip

\parbox{0.49\linewidth}{La figure ci-contre est dessinée à main levée. On donne les informations suivantes :

\begin{itemize}[label=\textbullet]
\item ABC est un triangle tel que :
AC = 10,4 cm, AB = 4 cm et BC = 9,6 cm ;
\item les points A, L et C sont alignés ;
\item les points B, K et C sont alignés ;
\item la droite (KL) est parallèle à la droite (AB) ;
\item CK = 3~cm.
\end{itemize}}\hfill
\parbox{0.49\linewidth}{\psset{unit=1cm}
\begin{pspicture}(6,2.8)
%\psgrid
\pslineByHand(0.5,0.5)(5.5,1)(4.75,2.5)(0.5,0.5)%CBA
\uput[ur](4.75,2.5){A} \uput[d](5.5,1){B} \uput[l](0.5,0.5){C} \uput[ul](1.9,1.1){L} \uput[dl](2.2,0.65){K} 
\pslineByHand(2.5,0)(1.5,2)%LK

\end{pspicture}
}
\medskip

\begin{enumerate}
\item À l'aide d'instruments de géométrie, construire la figure en vraie grandeur sur la copie en laissant apparents les traits de construction.
\item Prouver que le triangle ABC est rectangle en B.
\item Calculer la longueur CL en cm.
\item À l'aide de la calculatrice, calculer une valeur approchée de la mesure de l'angle $\widehat{\text{CAB}}$, au degré près.
\end{enumerate}

\bigskip

\textbf{Exercice 2 \hfill 15 points}

\medskip

Cet exercice est un questionnaire à choix multiple (QCM).

Pour chacune des cinq questions, quatre réponses sont proposées, une seule d'entre elles est exacte.

Pour chacune des cinq questions, indiquer sur la copie le numéro de la question et la réponse choisie .
 
\textbf{On rappelle que toute réponse doit être justifiée}.

Une réponse fausse ou l'absence de réponse ne retire pas de point.

\begin{center}
\begin{tabularx}{\linewidth}{|c m{4cm}|*{4}{>{\centering \arraybackslash}X|}}\hline
&Question&Réponse A &Réponse B &Réponse C &Réponse D\\ \hline
\textbf{1.}&Si on multiplie la longueur de chaque arête
 d'un cube par 3, alors le volume du cube sera multiplié par:&3 &9 &12 &27\\ \hline
\textbf{2.}&Lorsque $x$ est égal à $-4$,\: $x^2 +3x + 4$ est égal à :&8 &0 &$-24$ &$-13$\\ \hline
\textbf{3.}&$\dfrac{1}{3} + \dfrac{1}{4} = $&$\dfrac{2}{7}$&0,583&$\dfrac{7}{12}$&$\dfrac{1}{7}$\\ \hline
\textbf{4.}&La notation scientifique de \np{1500000000} est &$15 \times 10^{-8}$& $15 \times 10^8$&
$1,5 \times 10^{-9}$& $1,5 \times 10^9$\\ \hline
\textbf{5.}&$(x - 2)\times (x + 2)$	&$x^2 - 4$&	$x^2 +4$	&$2x - 4$ 	&$2x$\\ \hline
\end{tabularx}
\end{center}

\bigskip

\textbf{Exercice 3 \hfill 18 points}

\medskip

Dans cet exercice, le carré ABCD n'est pas représenté en vraie grandeur.

Aucune justification n'est attendue pour les questions 1. et 2. On attend des réponses justifiées pour la question 3.

\medskip

\begin{enumerate}
\item ~

\parbox{0.68\linewidth}{On considère le carré ABCD de centre O représenté ci-contre, partagé en quatre polygones superposables, numérotés \textcircled{1}, \textcircled{2}, \textcircled{3}, et \textcircled{4}.
	\begin{enumerate}
		\item Quelle est l'image du polygone \textcircled{1} par la symétrie centrale de centre O ?
		\item Quelle est l'image du polygone \textcircled{4} par la rotation de centre O qui transforme le polygone \textcircled{1} en le polygone \textcircled{2} ?
	\end{enumerate}}\hfill
\parbox{0.28\linewidth}{
\psset{unit=1cm}
\begin{pspicture}(-0.1,-0.1)(3,3)
\def\motif{\psline(0,1.4)(0,1.05)(-0.6,0.7)(0,0.35)(0,0)(0.35,0)(0.7,0.6)(1.05,0)(1.4,0)}
\psframe(0,0)(2.8,2.8)
\uput[ul](0,2.8){A}\uput[ur](2.8,2.8){B}\uput[dr](2.8,0){C}\uput[dl](0,0){D}
\rput(1.4,1.4){\motif}
\rput{90}(1.4,1.4){\motif}
\rput{180}(1.4,1.4){\motif}
\rput(0.5,2.2){\textcircled{1}}\rput(2.2,2.2){\textcircled{2}}
\rput(2.2,0.5){\textcircled{3}}\rput(0.5,0.5){\textcircled{4}}
\uput[dr](1.4,1.4){O}
\end{pspicture}
}
	
\item La figure ci-dessous est une partie de pavage dont un motif de base est le carré ABCD de la question 1.

Quelle transformation partant du polygone \textcircled{1} permet d'obtenir le polygone \textcircled{5} ?

\def\zig{\psset{unit=0.8cm}
\def\motif{\psline(0,1.4)(0,1.05)(-0.6,0.7)(0,0.35)(0,0)(0.35,0)(0.7,0.6)(1.05,0)(1.4,0)}
\rput(1.4,1.4){\motif}
\rput{90}(1.4,1.4){\motif}
\rput{180}(1.4,1.4){\motif}
}
\begin{center}
\psset{unit=0.8cm}
\begin{pspicture}(14,8.4)
\psframe(14,8.4)
%\psgrid
\multido{\n=0.0+2.8}{5}{
	\multido{\na=0.0+2.8}{3}{
	\rput(\n,\na){\zig}
	}
	}
\multido{\n=0.0+2.8}{6}
{\psline(\n,0)(\n,8.4)}
\multido{\n=0.0+2.8}{4}
{\psline(0,\n)(14,\n)}
\rput(0.6,7.7){\textcircled{1}}\rput(3.4,7.7){\textcircled{5}}
\uput[ul](0,8.4){A}\uput[ur](2.8,8.4){B}\uput[dl](0,5.6){D}\uput[dr](2.8,5.6){C}
\uput[dr](1.4,7){O}
\end{pspicture}
\end{center}
\item On souhaite faire imprimer ces motifs sur un tissu rectangulaire de longueur $315$ cm et de largeur $270$ cm.

On souhaite que le tissu soit entièrement recouvert par les carrés identiques à ABCD, sans découpe et de sorte que le côté du carré mesure un nombre entier de centimètres.
	\begin{enumerate}
		\item Montrer qu'on peut choisir des carrés de 9~cm de côté.
		\item Dans ce cas, combien de carrés de 9~cm de côté seront imprimés sur le tissu?
	\end{enumerate}
\end{enumerate}

\bigskip

\textbf{Exercice 4 \hfill 24 points}

\medskip

Voici la série des temps exprimés en secondes, et réalisés par des nageuses lors de la finale du 100 mètres féminin nage libre lors des championnats d'Europe de natation de 2018 :

\begin{center}
\begin{tabularx}{\linewidth}{|*{8}{>{\centering \arraybackslash}X|}}\hline
53,23&54,04&53,61&54,52&53,35&52,93&54,56&54,07\\ \hline
\end{tabularx}
\end{center}

\begin{enumerate}
\item La nageuse française, Charlotte BONNET, est arrivée troisième à cette finale. Quel est le temps, exprimé en secondes, de cette nageuse ?
\item Quelle est la vitesse moyenne, exprimée en m/s, de la nageuse ayant parcouru les $100$ mètres en $52,93$ secondes ? Arrondir au dixième près.
\item Comparer moyenne et médiane des temps de cette série.

\medskip

Sur une feuille de calcul, on a reporté le classement des dix premiers pays selon le nombre de médailles d'or lors de ces championnats d'Europe de natation, toutes disciplines confondues :

\begin{center}
\begin{tabularx}{\linewidth}{|c|c|c|*{4}{>{\centering \arraybackslash}X|}}\hline
&A&B &C& D &E &F\\ \hline
1& Rang &Nation &Or 	&Argent &Bronze &Total\\ \hline
2&1		&Russie 		&23	&15 &9 	&47 \\ \hline
3&2		&Grande-Bretagne&13	&12 &9 	&34 \\ \hline
4&3		&Italie 		&8	&12 &19 &39\\ \hline
5&4		&Hongrie 		&6	&4	&2	&12\\ \hline
6&5		&Ukraine 		&5	&6 	&2 	&13\\ \hline
7&6		&Pays-Bas 		&5	&5 	&2 	&12\\ \hline
8&7		&France 		&4	&2 	&6 	&12\\ \hline
9&8		&Suède 			&4	&0 	&0 	&4\\ \hline
10&9	&Allemagne 		&3	&6 	&10 &19\\ \hline
11&10	&Suisse 		&1	&0 	&1 	&2\\ \hline
\end{tabularx}
\end{center}

\item Est-il vrai qu'à elles deux, la Grande-Bretagne et l'Italie ont obtenu autant de médailles d'or
que la Russie ?
\item Est-il vrai que plus de 35\,\% des médailles remportées par la France sont des médailles d'or ?
\item Quelle formule a-t-on pu saisir dans la cellule F2 de cette feuille de calcul, avant qu'elle soit
étirée vers le bas jusqu'à la cellule F11 ?

\end{enumerate}

\bigskip

\textbf{Exercice 5 \hfill 23 points}

\medskip

On dispose de deux urnes:

\begin{itemize}
\item une urne bleue contenant trois boules bleues numérotées: \textcircled{2},\textcircled{3} et \textcircled{4}.
\item une urne rouge contenant quatre boules rouges numérotées: \textcircled{2},\textcircled{3}, \textcircled{4} et \textcircled{5}.
\end{itemize}

Dans chaque urne, les boules sont indiscernables au toucher et ont la même probabilité d'être tirées.

\begin{center}
\begin{tabularx}{\linewidth}{|*{2}{>{\centering \arraybackslash}X|}}\hline
Urne bleue &Urne rouge\\
\textcircled{2} \textcircled{3}  \textcircled{4}& \textcircled{2} \textcircled{3} \textcircled{4}  \textcircled{5}\\ \hline
\end{tabularx}
\end{center}
\medskip

On s'intéresse à l'expérience aléatoire suivante :

\og On tire au hasard une boule bleue et on note son numéro, puis on tire au hasard une boule rouge et on note son numéro. \fg

\emph{Exemple} : si on tire la boule bleue numérotée \textcircled{3}, puis la boule rouge numérotée \textcircled{4},le tirage obtenu sera noté (3~;~4).

On précise que le tirage (3~;~4) est différent du tirage (4~;~3).

\medskip

\begin{enumerate}
\item On définit les deux évènements suivants:

\og On obtient deux nombres premiers \fg{} et \og La somme des deux nombres est égale à 12 \fg
	\begin{enumerate}
		\item Pour chacun des deux évènements précédents, dire s'il est possible ou impossible lorsqu'on effectue l'expérience aléatoire.
		\item Déterminer la probabilité de l'évènement \og On obtient deux nombres premiers \fg.
	\end{enumerate}
\item On obtient un \og double \fg{} lorsque les deux boules tirées portent le même numéro. 

Justifier que la probabilité d'obtenir un \og double \fg{} lors de cette expérience, est $\dfrac{1}{4}$.
\item Dans cette question, aucune justification n'est attendue. 

On souhaite simuler cette expérience \np{1000} fois.

Pour cela, on a commencé à écrire un programme, à ce stade, encore incomplet. Voici des copies d'écran :

\begin{center}
{\footnotesize
\begin{tabular}{|c |c|}\hline
\textbf{Script principal}& \textbf{Bloc \og Tirer deux boules \fg}\\
\begin{scratch}
\blockinit{quand \greenflag est cliqué},
\blockrepeat{répéter \ovalnum{A} fois}
		{
		\blockif{si \booloperator{\ovalmove{Boule bleue} = \ovalmove{Boule rouge}} alors}
{\blockmove{ajouter à \selectmenu{Nombre de doubles} \ovalnum{1}}}		
		}	
\end{scratch}
&
\begin{scratch}	
\initmoreblocks{définir \namemoreblocks{Tirer deux boules}}
\blockmove{mettre \selectmenu{Boule bleue} à {nombre aléatoire entre \ovalnum{2} et \ovalnum{B}}}
\blockmove{mettre \selectmenu{Boule rouge} à {nombre aléatoire entre \ovalnum{2} et \ovalnum{C}}}
\end{scratch}
\\ 
\multicolumn{2}{|c|}{Boule bleue, Boule rouge et Nombre de doubles 
 sont des variables.}\\
\multicolumn{2}{|c|}{Le bloc \begin{scratch}\blockmove{Tirer deux boules}\end{scratch} est à insérer dans le script principal.}\\ \hline
\end{tabular}}
\end{center}
	\begin{enumerate}
		\item Par quels nombres faut-il remplacer les lettres A, B et C ?
		\item Dans le script principal, indiquer où placer le bloc \begin{scratch}\blockmove{Tirer deux boules}\end{scratch}
		\item Dans le script principal, indiquer où placer le bloc 
\begin{scratch}
\blockmove{mettre \selectmenu{Nombre de doubles}à \ovalnum{0} }
\end{scratch}
		\item On souhaite obtenir la fréquence d'apparition du nombre de \og doubles \fg{} obtenus.
		
Parmi les instructions ci-dessous, laquelle faut-il placer à la fin du script principal après la
boucle \og répéter \fg{} ?

\begin{center}
{\footnotesize
\begin{tabular}{|ccc|}\hline
Proposition \textcircled{1}&Proposition \textcircled{2} & Proposition \textcircled{3}\\
\begin{scratch}
\blockmove{dire \ovaloperator{Nombre de doubles}} 
\end{scratch}&
\begin{scratch} 
\blockmove{dire \ovaloperator{Nombre de doubles}/\ovalnum{1000}}\end{scratch}&\begin{scratch} 
\blockmove{dire \ovaloperator{Nombre de doubles}/\ovalnum{2}}
\end{scratch}\\ \hline
\end{tabular}}
\end{center}
	\end{enumerate}
\end{enumerate}
\end{document}