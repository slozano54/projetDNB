\documentclass[10pt]{article}
\usepackage[T1]{fontenc}
\usepackage[utf8]{inputenc}%ATTENTION codage UTF8
\usepackage{fourier}
\usepackage[scaled=0.875]{helvet}
\renewcommand{\ttdefault}{lmtt}
\usepackage{amsmath,amssymb,makeidx}
\usepackage[normalem]{ulem}
\usepackage{diagbox}
\usepackage{fancybox}
\usepackage{tabularx}
\usepackage{ulem}
\usepackage{pifont}
\renewcommand{\sfdefault}{phv}% police helvetica pour les blocs scratch.
\usepackage{multirow}
\usepackage{dcolumn}
\usepackage{textcomp}
\usepackage{lscape}
\newcommand{\euro}{\eurologo{}}
\usepackage{graphics,graphicx}
\usepackage{pstricks,pst-plot,pst-tree,pstricks-add}
\usepackage[left=3.5cm, right=3.5cm, top=3cm, bottom=3cm]{geometry}
\newcommand{\R}{\mathbb{R}}
\newcommand{\N}{\mathbb{N}}
\newcommand{\D}{\mathbb{D}}
\newcommand{\Z}{\mathbb{Z}}
\newcommand{\Q}{\mathbb{Q}}
\newcommand{\C}{\mathbb{C}}
\usepackage{libertine,boites,tikz,xspace,scratch}
\usepackage{scratch}
\renewcommand{\theenumi}{\textbf{\arabic{enumi}}}
\renewcommand{\labelenumi}{\textbf{\theenumi.}}
\renewcommand{\theenumii}{\textbf{\alph{enumii}}}
\renewcommand{\labelenumii}{\textbf{\theenumii.}}
\newcommand{\vect}[1]{\overrightarrow{\,\mathstrut#1\,}}
\def\Oij{$\left(\text{O}~:~\vect{\imath},~\vect{\jmath}\right)$}
\def\Oijk{$\left(\text{O}~:~\vect{\imath},~\vect{\jmath},~\vect{k}\right)$}
\def\Ouv{$\left(\text{O}~:~\vect{u},~\vect{v}\right)$}
\usepackage{fancyhdr}
\usepackage[dvips]{hyperref}
\hypersetup{%
pdfauthor = {APMEP},
pdfsubject = {Brevet des collèges},
pdftitle = {Polynésie 10 septembre 2018},
allbordercolors = white,
pdfstartview=FitH}   
\usepackage[frenchb]{babel}
\usepackage[np]{numprint}
\begin{document}
\setlength\parindent{0mm}
\rhead{\textbf{A. P{}. M. E. P{}.}}
\lhead{\small Brevet des collèges}
\lfoot{\small{Polynésie}}
\rfoot{\small{10 septembre 2018}}
\pagestyle{fancy}
\thispagestyle{empty}
\begin{center}
 
{\Large \textbf{\decofourleft~Brevet des collèges Polynésie 10 septembre 2018~\decofourright}}

\bigskip

\textbf{Durée : 2 heures} \end{center}

\bigskip

\begin{tabularx}{\linewidth}{|X|}\hline
\multicolumn{1}{|c|}{Indications portant sur \textbf{l'ensemble du sujet}}\\
Toutes les réponses doivent être justifiées, sauf si une indication contraire est
donnée.\\
Pour chaque question, si le travail n'est pas terminé, laisser tout de même une
trace de la recherche; elle sera prise en compte dans la notation.\\ \hline
\end{tabularx}

\bigskip

\textbf{Exercice 1 \hfill 12 points}

\medskip

Indiquer si les affirmations suivantes sont vraies ou fausses. Justifier vos réponses.

\medskip

\textbf{Affirmation 1}

\smallskip

On lance un dé équilibré à six faces numérotées de 1 à 6.

Un élève affirme qu'il a deux chances sur trois d'obtenir un diviseur de 6.

A-t-il raison ?

\medskip

\textbf{Affirmation 2}

\smallskip

On considère le nombre $a = 3^4 \times 7$.

Un élève affirme que le nombre $b = 2 \times 3^5 \times 7^2$ est un multiple du nombre $a$.

A-t-il raison ?

\medskip

\textbf{Affirmation 3}

\smallskip

En 2016, le football féminin comptait en France \np{98800} licenciées alors qu'il y en avait \np{76000} en 2014.

Un journaliste affirme que le nombre de licenciées a augmenté de $30$\,\% de 2014 à
2016.

A-t-il raison ?

\medskip

\textbf{Affirmation 4}

\smallskip

Une personne A a acheté un pull et un pantalon de jogging dans un magasin.

Le pantalon de jogging coûtait 54~\euro. Dans ce magasin, une personne B a acheté le
même pull en trois exemplaires; elle a dépensé plus d'argent que la personne A.

La personne B affirme qu'un pull coûte $25$~\euro.

A-t-elle raison ?

\bigskip

\textbf{Exercice 2 \hfill 14 points}

\medskip

Un amateur de football, après l'Euro 2016, décide de s'intéresser à l'historique des
treize dernières rencontres entre la France et le Portugal, regroupées dans le tableau
ci-dessous.

On rappelle la signification des résultats ci-dessous en commentant deux exemples :

\setlength\parindent{6mm}
\begin{itemize}
\item[$\bullet~~$]la rencontre du 3 mars 1973, qui s'est déroulée en France, a vu la victoire du Portugal par 2 buts à 1 ;
\item[$\bullet~~$]la rencontre du 8 mars 1978, qui s'est déroulée en France, a vu la victoire de la France par 2 buts à 0.
\end{itemize}
\setlength\parindent{0mm}

\begin{center}
\begin{tabularx}{\linewidth}{|*{3}{>{\centering \arraybackslash}X|}}
\hline
\multicolumn{3}{|c|}{\textbf{Rencontres de football opposant la France et le Portugal depuis 1973}}\\ \hline
3 mars 1973		& France - Portugal& 1-2\\ \hline
26 avril 1975	& France - Portugal& 0-2\\ \hline
8 mars 1978		& France - Portugal& 2-0\\ \hline
16 février 1983	& Portugal - France& 0-3\\ \hline
23 juin 1984	& France - Portugal& 3-2\\ \hline
24 janvier 1996	& France - Portugal& 3-2\\ \hline
22 janvier 1997	& Portugal - France& 0-2\\ \hline
28 juin 2000	& Portugal - France& 1-2\\ \hline
25 avril 2001	& France - Portugal& 4-0\\ \hline
5 juillet 2006	& Portugal - France& 0-1\\ \hline
11 octobre 2014	& France - Portugal& 2-1\\ \hline
4 septembre 2015& Portugal - France& 0-1\\ \hline
10 juillet 2016	& France - Portugal& 0-1\\ \hline
\end{tabularx}
\end{center}

\begin{enumerate}
\item Depuis 1973, combien de fois la France a-t-elle gagné contre le Portugal ?
\item Calculer le pourcentage du nombre de victoires de la France contre le Portugal
depuis 1973. Arrondir le résultat à l'unité de \%.
\item Le 3 mars 1973, 3 buts ont été marqués au cours du match. Calculer le nombre
moyen de buts par match sur l'ensemble des rencontres. Arrondir le résultat au
dixième.
\end{enumerate}

\bigskip

\textbf{Exercice 3 \hfill 16 points}

\medskip

Une personne s'intéresse à un magazine sportif qui parait une fois par semaine. Elle
étudie plusieurs formules d'achat de ces magazines qui sont détaillées ci-après.

\begin{center}
\begin{tabularx}{\linewidth}{|X|}\hline
$\bullet~~$ Formule A - Prix du magazine à l'unité: 3,75~\euro{} ;\\
$\bullet~~$ Formule B - Abonnement pour l'année: 130~\euro{} ;\\
$\bullet~~$ Formule C - Forfait de 30~\euro{} pour l'année et 2,25~\euro{} par magazine.\\ \hline
\end{tabularx}
\end{center}

On donne ci-dessous les représentations graphiques qui correspondent à ces trois
formules.

\begin{center}
\psset{xunit=0.225cm,yunit=0.05cm}
\begin{pspicture}(-1,-5)(49,150)
\multido{\n=0+2}{25}{\psline[linewidth=0.3pt](\n,0)(\n,150)}
\multido{\n=0+20}{8}{\psline[linewidth=0.3pt](0,\n)(49,\n)}
\psaxes[linewidth=1.25pt,Dx=2,Dy=20]{->}(0,0)(0,0)(49,150)
\psaxes[linewidth=1.25pt,Dx=2,Dy=20](0,0)(0,0)(49,150)
\uput*[u](41,0){Nombre de magazines}
\uput*[r](0,145){Coût en euros}
\psline(40,150)
\psline(0,130)(49,130)
\psplot[plotpoints=3000,linewidth=1.25pt]{0}{48}{2.25 x mul 30 add}
\uput[d](30,95){(D$_1$)} \uput[u](30,130){(D$_2$)} \uput[u](30,112){(D$_3$)} 
\end{pspicture}
\end{center}

\begin{enumerate}
\item Sur votre copie, recopier le contenu du cadre ci-dessous et relier par un trait
chaque formule d'achat avec sa représentation graphique.

\begin{center}
\newcolumntype{Y}{>{\raggedleft \arraybackslash}X}
\begin{tabularx}{0.6\linewidth}{|X Y|}\hline
Formule A  \psdots[dotstyle=+,dotangle=45](0.1,0.1)&\psdots[dotstyle=+,dotangle=45](0,0.1)~(D1)
\\
Formule B \psdots[dotstyle=+,dotangle=45](0.1,0.1)&\psdots[dotstyle=+,dotangle=45](0,0.1)~(D2)\\
Formule C \psdots[dotstyle=+,dotangle=45](0.1,0.1)&\psdots[dotstyle=+,dotangle=45](0,0.1)~(D3)\\ \hline
\end{tabularx}
\end{center}

\item  En utilisant le graphique, répondre aux questions suivantes.

\emph{Les traits de construction devront apparaitre sur le graphique en ANNEXE qui est à rendre avec la copie.}
	\begin{enumerate}
		\item En choisissant la formule A, quelle somme dépense-t-on pour acheter
16 magazines dans l'année ?
		\item Avec $120$~\euro, combien peut-on acheter de magazines au maximum dans
une année avec la formule C ?
		\item Si on décide de ne pas dépasser un budget de $100$~\euro{} pour l'année, quelle est alors la formule qui permet d'acheter le plus grand nombre de
magazines ?
	\end{enumerate}
\item  Indiquer la formule la plus avantageuse selon le nombre de magazines achetés
dans l'année.
\end{enumerate}

\bigskip

\textbf{Exercice 4 \hfill 14 points}

\medskip

Un garçon et une fille pratiquent le roller. Ils décident de faire une course en
empruntant deux parcours différents. 

La fille, qui part du point F et arrive au point A, met 28,5 secondes. 

Le garçon, qui part du point G et arrive aussi au point A, met 28
secondes.

Le dessin ci-après, qui n'est pas à l'échelle, représente les deux parcours; celui de la
fille comporte deux demi-cercles de $5$~m de rayon.

\begin{center}
\psset{unit=1cm}
\begin{pspicture}(12,3)
%\psgrid
\psline(0,2.5)(10,2.5)
\psarc(10,2){0.5}{-90}{90}
\psline(10,1.5)(8,1.5)
\psarc(8,1){0.5}{90}{270}
\psline(8,0.5)(10,0.5)
\uput[u](4,2.5){200 m} \uput[l](0,2.5){G}\uput[u](8,2.5){A} 
\uput[r](8,1.25){5 m}\uput[u](10,2.5){B}\uput[d](10,1.5){C}
\uput[u](8,1.5){D}\uput[d](8,0.5){E}\uput[r](10,0.5){F}
\uput[u](8,2.5){A}\uput[u](9,2.5){60 m}
\psdots[dotstyle=+,dotangle=45](8,2.5)(10,2.5)(10,1.5)(8,1.5)(8,0.5)(10,0.5)(0,2.5)
\rput(9,2.5){|||}\rput(9,1.5){|||}\rput(9,0.5){|||}
\psline{->}(8,1)(8,1.5)\psline{->}(10,2)(10,2.5)
\uput[r](10,2.25){5 m}
\end{pspicture}
\end{center}

\begin{enumerate}
\item Quel est le parcours le plus long ?
\item Qui se déplace le plus vite, le garçon ou la fille ?
\end{enumerate}
\smallskip

\emph{On rappelle que si $p$ est le périmètre d'un cercle de rayon $r$, alors} $p = 2 \times \pi \times r$.

\bigskip

\textbf{Exercice 5 \hfill 14 points}

\medskip

Un collégien français et son correspondant anglais ont de nombreux
centres d'intérêt communs comme le basket qu'ils pratiquent tous les deux.

Le tableau ci-dessous donne quelques informations sur leurs ballons.

\begin{center}
\begin{tabularx}{\linewidth}{|X|X|}\hline
\multicolumn{1}{|c|}{Ballon du collégien français}& \multicolumn{1}{|c|}{Ballon du correspondant anglais}\\ \hline
\multicolumn{1}{|c|}{$A \approx \np{1950}$ cm"}& \multicolumn{1}{|c|}{$D \approx 9,5$ inch}\\ \hline
&$D$ désigne le diamètre du ballon.\\
$A$ désigne l'aire de la surface du ballon et $r$ son rayon. On a $A = 4 \times \pi \times r^2$.& L'inch est une unité de longueur anglo-saxonne.
On a 1 inch $= 2,54$ cm.\\ \hline
\end{tabularx}
\end{center}

Pour qu'un ballon soit utilisé dans un match officiel, son diamètre doit être compris
entre $23,8$ cm et $24,8$ cm.

\medskip

\begin{enumerate}
\item Le ballon du collégien français respecte-t-il cette norme ?
\item  Le ballon du collégien anglais respecte-t-il cette norme ?
\end{enumerate}

\bigskip

\textbf{Exercice 6 \hfill 12 points}

\medskip

Une personne pratique le vélo de piscine depuis plusieurs années dans un centre
aquatique à raison de deux séances par semaine. Possédant une piscine depuis peu,
elle envisage d'acheter un vélo de piscine pour pouvoir l'utiliser exclusivement chez
elle et ainsi ne plus se rendre au centre aquatique.

\medskip

\setlength\parindent{6mm}
\begin{itemize}
\item[$\bullet~~$] Prix de la séance au centre aquatique: 15~\euro.
\item[$\bullet~~$] Prix d'achat d'un vélo de piscine pour une pratique à la maison: 999~\euro.
\end{itemize}
\setlength\parindent{0mm}

\medskip

\begin{enumerate}
\item Montrer que 10 semaines de séances au centre aquatique lui coûtent 300~\euro.
\item Que représente la solution affichée par le programme ci-après?

\begin{center}
\begin{scratch}
\blockinit{quand \greenflag est cliqué}
\blockvariable{mettre \selectmenu{x} à \txtbox{0}}
\blockinfloop{répéter jusqu'à \booloperator{\ovalvariable{x} * 2 * 15 > \txtbox{999}}}
{\blockvariable{ajouter à \selectmenu{x} \ovalnum{1}}
}
\blocklook{dire \ovaloperator{regroupe}{\txtbox{La solution est :}}\ovalvariable{x}}

\end{scratch}
\end{center}

\item  Combien de semaines faudrait-il pour que l'achat du vélo de piscine soit
rentabilisé?
\end{enumerate}

\bigskip

\textbf{Exercice 7 \hfill 18 points}

\medskip

\textbf{1\up{re} partie}

\smallskip

\parbox{0.45\linewidth}{\psset{unit=1cm}
\begin{pspicture}(7.4,3.6)
%\psgrid
\psline[linewidth=1.5pt](0,1.3)(5.3,1.3)(7,1.6)(7.4,1.8)
\psline[linewidth=1.5pt,linestyle=dashed](6.9,1.9)(7.4,1.8)
\psline[linewidth=1.5pt](0.6,1.7)(1.2,2)(6.5,2)(6.9,1.9)
\psline[linewidth=1.5pt,linestyle=dashed](0,1.3)(0.6,1.7)
\pspolygon(0,1.65)(5.3,1.65)(7.1,1.9)(7.4,2)(6.5,2.3)(1.15,2.3)
\psline(0,0)(5.3,0)(7.1,0.2)(7.4,0.5)
\psline[linestyle=dashed](7.4,0.5)(6.4,1.1)(1.2,1.1)(0,0)
\psline[linestyle=dashed](6.4,1.2)(6.4,2.3)
\psline[linestyle=dashed](1.2,1.1)(1.2,2)
\psline(1.2,2)(1.2,2.3)
\psline(0,1.7)(0,0)\psline(5.3,0)(5.3,1.7)\psline(7.1,0.2)(7.1,1.9)
\psline(7.4,0.5)(7.4,2)
\rput(6,3.3){Frise de la piscine}\psline{->}(6,3.1)(5.5,2.2)
\end{pspicture}}\hfill
\parbox{0.45\linewidth}{Une personne possède une piscine. 

Elle veut coller une frise en carrelage au niveau de la ligne d'eau.}

\medskip

La piscine vue de haut, est représentée à l'échelle par la partie grisée du schéma ci-après.

\begin{center}
\psset{unit=1cm}
\begin{pspicture}(10,4.5)
\pspolygon[linewidth=1.3pt,fillstyle=solid,fillcolor=lightgray](0.5,0.5)(7.5,0.5)(9.1,1.8)(9.1,2.6)(7.5,4)(0.5,4)%HGEDBA
\psline(7.5,0.5)(9.1,0.5)(9.1,4)(7.5,4)
\uput[dl](0.5,0.5){H}\uput[d](7.5,0.5){G}\uput[r](9.1,1.8){E}\uput[r](9.1,2.6){D}
\uput[u](7.5,4){B}\uput[ul](0.5,4){A}\uput[dr](9.1,0.5){F}\uput[ur](9.1,4){C}
\rput(8.3,4){|}\rput(8.3,0.5){|}\rput{90}(9.1,3.3){|}\rput{90}(9.1,3.4){|}
\rput{90}(9.1,1){|}\rput{90}(9.1,1.1){|}
\end{pspicture}
\end{center}


\textbf{Données :}

\smallskip

\setlength\parindent{6mm}
\begin{itemize}
\item[$\bullet~~$]le quadrilatère ACFH est un rectangle;
\item[$\bullet~~$]le point B est sur le côté [AC] et le point G est sur le côté [FH] ;
\item[$\bullet~~$]les points D et E sont sur le côté [CF] ;
\item[$\bullet~~$]AC $= 10$ m; AH $= 4$ m ; BC = FG $= 2$ m ; CD = EF $= 1,5$ m.
\end{itemize}
\setlength\parindent{0mm}

\smallskip

\textbf{Question :}

Calculer la longueur de la frise.

\bigskip

\textbf{2\up{e} partie}

\smallskip

La personne décide d'installer, au-dessus de la piscine, une grande voile d'ombrage
qui se compose de deux parties détachables reliées par une fermeture éclair comme
le montre le schéma ci-dessous qui n'est pas à l'échelle.

\begin{center}
\psset{unit=1cm}
\begin{pspicture}(12,9)
\pspolygon[fillstyle=solid,fillcolor=lightgray](0,4.5)(6.4,4.5)(8,5.5)(8,7)(6.4,8)(0,8)
\pspolygon[fillstyle=solid,fillcolor=white](0.5,0.5)(8.4,1)(1,8.5)
\psline[linestyle=dashed,linewidth=1.5pt](0.66,3.9)(5.5,4)
\uput[dl](0.5,0.5){M}\uput[dr](8.4,1){O}\uput[u](1,8.5){K}
\uput[l](0.65,3.9){L}\uput[r](5.7,4){N}
\rput(9.5,7.5){Piscine}\psline{->}(8.8,7.4)(7.5,6)
\rput(9,3){2 parties du voile d'ombrage}
\psline{->}(6.8,3)(5.7,1.5)
\psline{->}(6.8,3.2)(3,5)
\end{pspicture}
\end{center}

\textbf{Données :}

\smallskip

\setlength\parindent{6mm}
\begin{itemize}
\item[$\bullet~~$]la première partie couvrant une partie de la piscine est représentée par le
triangle KLN ;
\item[$\bullet~~$]la deuxième partie est représentée par le trapèze LMON de bases [LN] et [MO] ;
\item[$\bullet~~$]la fermeture éclair est représentée par le segment [LN] ;
\item[$\bullet~~$]les poteaux, soutenant la voile d'ombrage positionnés sur les points K, L et M, sont alignés;
\item[$\bullet~~$]les poteaux, soutenant la voile d'ombrage positionnés sur les points K, N et O,
sont alignés;
\item[$\bullet~~$]KL $= 5$ m ; LM $= 3,5$ m ; NO $= 5,25$ m ; MO $= 10,2$ m.
\end{itemize}
\setlength\parindent{0mm}

\smallskip

\textbf{Question :}

Calculer la longueur de la fermeture éclair.
\bigskip

\newpage

\begin{center}

\textbf{\large ANNEXE}

\medskip

\textbf{À détacher du sujet et à joindre avec la copie}

\vspace{1cm}

\textbf{Exercice 3 question 2}

\medskip

\psset{xunit=0.225cm,yunit=0.05cm}
\begin{pspicture}(-1,-5)(49,150)
\multido{\n=0+2}{25}{\psline[linewidth=0.3pt](\n,0)(\n,150)}
\multido{\n=0+20}{8}{\psline[linewidth=0.3pt](0,\n)(49,\n)}
\psaxes[linewidth=1.25pt,Dx=2,Dy=20]{->}(0,0)(0,0)(49,150)
\psaxes[linewidth=1.25pt,Dx=2,Dy=20](0,0)(0,0)(49,150)
\uput*[u](41,0){Nombre de magazines}
\uput*[r](0,145){Coût en euros}
\psline(40,150)
\psline(0,130)(49,130)
\psplot[plotpoints=3000,linewidth=1.25pt]{0}{48}{2.25 x mul 30 add}
\uput[d](30,95){(D$_1$)} \uput[u](30,130){(D$_2$)} \uput[u](30,112){(D$_3$)} 
\end{pspicture}
\end{center}
\end{document}