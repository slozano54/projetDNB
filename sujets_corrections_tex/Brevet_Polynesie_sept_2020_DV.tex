\documentclass[10pt]{article}
\usepackage[T1]{fontenc}
\usepackage[utf8]{inputenc}%ATTENTION codage UTF8
\usepackage{fourier}
\usepackage[scaled=0.875]{helvet}
\renewcommand{\ttdefault}{lmtt}
\usepackage{amsmath,amssymb,makeidx}
\usepackage[normalem]{ulem}
\usepackage{diagbox}
\usepackage{fancybox}
\usepackage{tabularx,booktabs}
\usepackage{colortbl}
\usepackage{pifont}
\usepackage{multirow}
\usepackage{dcolumn}
\usepackage{enumitem}
\usepackage{textcomp}
\usepackage{lscape}
\newcommand{\euro}{\eurologo{}}
\usepackage{graphics,graphicx}
\usepackage{pstricks,pst-plot,pst-tree,pstricks-add}
\usepackage[left=3.5cm, right=3.5cm, top=3cm, bottom=3cm]{geometry}
\newcommand{\R}{\mathbb{R}}
\newcommand{\N}{\mathbb{N}}
\newcommand{\D}{\mathbb{D}}
\newcommand{\Z}{\mathbb{Z}}
\newcommand{\Q}{\mathbb{Q}}
\newcommand{\C}{\mathbb{C}}
\usepackage{scratch}
\renewcommand{\theenumi}{\textbf{\arabic{enumi}}}
\renewcommand{\labelenumi}{\textbf{\theenumi.}}
\renewcommand{\theenumii}{\textbf{\alph{enumii}}}
\renewcommand{\labelenumii}{\textbf{\theenumii.}}
\newcommand{\vect}[1]{\overrightarrow{\,\mathstrut#1\,}}
\def\Oij{$\left(\text{O}~;~\vect{\imath},~\vect{\jmath}\right)$}
\def\Oijk{$\left(\text{O}~;~\vect{\imath},~\vect{\jmath},~\vect{k}\right)$}
\def\Ouv{$\left(\text{O}~;~\vect{u},~\vect{v}\right)$}
\usepackage{fancyhdr}
\usepackage[french]{babel}
\usepackage[dvips]{hyperref}
\hypersetup{%
pdfauthor = {APMEP},
pdfsubject = {Brevet des collèges},
pdftitle = {Polynésie 7 
septembre 2020},
allbordercolors = white,
pdfstartview=FitH}   
\usepackage[np]{numprint}
%\frenchbsetup{StandardLists=true}
\begin{document}
\setlength\parindent{0mm}
\rhead{\textbf{A. P{}. M. E. P{}.}}
\lhead{\small Brevet des collèges}
\lfoot{\small{Polynésie}}
\rfoot{\small{7 septembre 2020}}
\pagestyle{fancy}
\thispagestyle{empty}
\begin{center}
    
{\Large \textbf{\decofourleft~Brevet des collèges Polynésie 7 septembre 2020~\decofourright}}
    
\bigskip
    
\textbf{Durée : 2 heures} \end{center}

\bigskip

\textbf{\begin{tabularx}{\linewidth}{|X|}\hline
 L'évaluation prend en compte la clarté et la précision des raisonnements ainsi que, plus largement, la qualité de la rédaction. Elle prend en compte les essais et les démarches engagées même non abouties. Toutes les réponses doivent être justifiées, sauf mention contraire.\\ \hline
\end{tabularx}}

\vspace{0.5cm}

\textbf{Exercice 1 \hfill 22 points}

\medskip

\emph{Dans cet exercice, toutes les questions sont indépendantes}

\medskip

\begin{enumerate}
\item ~

\parbox{0.6\linewidth}{Quel nombre obtient-on avec le programme de calcul ci- contre, si l'on choisit comme nombre de départ $-7$ ?}\hfill
\parbox{0.38\linewidth}{
\begin{tabular}{|l|}\hline
\multicolumn{1}{|c|}{\textbf{Programme de calcul}}\\
Choisir un nombre de départ.\\
Ajouter 2 au nombre de départ.\\
Élever au carré le résultat.\\ \hline
 \end{tabular}}
\item  Développer et réduire l'expression $(2x - 3)(4x + 1)$.
 
\item~

\parbox{0.6\linewidth}{Sur la figure ci-contre, qui n'est pas à l'échelle, les droites (AB) et (DE) sont parallèles.

Les points A, C et D sont alignés.

Les points B, C et E sont alignés.

Calculer la longueur CB.}\hfill
\parbox{0.38\linewidth}{\psset{unit=0.9cm}
\begin{pspicture}(5.5,5)
\psline(0.3,0)(5.4,1.6)
\psline(0,2.8)(5.3,4.6)
\psline(1.6,0.4)(5,4.5)%EB
\psline(0.6,3)(3.5,1)%AD
\uput[u](0.6,3){A} \uput[u](5,4.5){B} \uput[u](2.6,1.6){C} 
\uput[d](3.5,1){D} \uput[d](1.6,0.4){E}
\uput[l](2.2,1.2){1,5 cm}\uput[ur](3,1.3){1 cm}\uput[u](1.9,2.3){3,5 cm} 
\end{pspicture}}
\item Un article coûte 22~\euro. Son prix baisse de 15\,\%. Quel est son nouveau prix ?
\item  Les salaires mensuels des employés d'une entreprise sont présentés dans le tableau suivant.

\begin{center}
\begin{tabularx}{\linewidth}{|m{2.7cm}|*{7}{>{\centering \arraybackslash}X|}}\hline
Salaire mensuel (en euro)&\np{1300} &\np{1400} &\np{1500} &\np{1900} &\np{2000} &\np{2700} &\np{3500}\\
 \hline
Effectif				 & 11 		&6 		&5 		&3 		&3 		&1 		&1\\ \hline
\end{tabularx}
\end{center}

Déterminer le salaire médian et l'étendue des salaires dans cette entreprise.
\item Quel est le plus grand nombre premier qui divise \np{41895} ?
\end{enumerate}

\vspace{0.5cm}

\textbf{Exercice 2 \hfill 15 points}

\medskip

On souhaite réaliser une frise composée de rectangles. 

Pour cela, on a écrit le programme ci-dessous:

\begin{center}
\begin{tabularx}{\linewidth}{|X|X|}\hline
\begin{scratch}
\blockinit{quand \greenflag est cliqué}
\blockcontrol{cacher}
\blockpen{mettre la taille du stylo à \ovalnum{1}}
\blockpen{effacer tout}
\blockmove{aller à x: \ovalnum0 y: \ovalnum0}
\blockrepeat{répéter \ovalnum{5} fois}
{\blockmoreblocks{Rectangle}
\blockmove{ajouter \ovalnum{40} à \ovalvariable{x}}
\blockmove{ajouter \ovalnum{-20} à \ovalvariable{y}}
}
\end{scratch}&
\begin{scratch}
\initmoreblocks{définir \namemoreblocks{Rectangle}}
\blockpen{stylo en position d'écriture}
\blockmove{s’orienter à \ovalnum{90} degrés}
\blockrepeat{répéter \ovalnum{2} fois}
{\blockmove{avancer de \ovalnum{40}}
\blockmove{tourner \turnright{} de \ovalnum{90} degrés}
\blockmove{avancer de \ovalnum{20}}
\blockmove{tourner \turnright{} de \ovalnum{90} degrés}}
\blockpen{relever le stylo}
\end{scratch}\\
\textbf{Script principal} &\textbf{Bloc \og rectangle\fg}\\ \hline
\end{tabularx}
\end{center}

On rappelle que l'instruction \og s'orienter à 90 \fg{} consiste à s'orienter horizontalement vers la droite. 

\medskip

\textbf{Dans cet exercice, aucune justification n'est demandée}

\medskip

\begin{enumerate}
\item Quelles sont les coordonnées du point de départ du tracé ?
\item Combien de rectangles sont dessinés par le script principal?
\item Dessiner à main levée la figure obtenue avec le script principal.
\item 
	\begin{enumerate}
		\item Sans modifier le script principal, on a obtenu la figure ci-dessous composée de rectangles de longueur $40$ pixels et de largeur $20$ pixels. Proposer une modification du bloc \og rectangle\fg permettant d'obtenir cette figure.

\begin{center}
\psset{unit=1cm}
\begin{pspicture}(4.5,2.8)
\multirput(0,2)(0.8,-0.4){5}{\psframe(0.4,0.8)}
\end{pspicture}
\end{center}
		\item Où peut-on alors ajouter l'instruction \begin{scratch}\blockmove{ajouter \ovalnum{1} à la taille du stylo}\end{scratch} dans le script principal pour obtenir la figure ci-dessous ?
		
\begin{center}
\psset{unit=1cm}
\begin{pspicture}(4.5,2.8)
\rput(0,2){\psframe[linewidth=1pt](0.4,0.8)}
\rput(0.8,1.6){\psframe[linewidth=1.5pt](0.4,0.8)}
\rput(1.6,1.2){\psframe[linewidth=2pt](0.4,0.8)}
\rput(2.4,0.8){\psframe[linewidth=2.5pt](0.4,0.8)}
\rput(3.2,0.4){\psframe[linewidth=3pt](0.4,0.8)}
\end{pspicture}
\end{center}		
	\end{enumerate}
\end{enumerate}

\vspace{0.5cm}

\textbf{Exercice 3 \hfill 26 points}

\medskip

\parbox{0.7\linewidth}{On considère le motif initial ci-contre.

Il est composé d'un carré ABCE de côté $5$~cm et d'un triangle EDC, rectangle et isocèle en D.}
\hfill
\parbox{0.28\linewidth}{\psset{unit=1cm}
\begin{pspicture}(3,4)
\pspolygon[fillstyle=solid,fillcolor=lightgray](0.5,0.5)(2.5,0.5)(2.5,2.5)(1.5,3.5)(0.5,2.5)%ABCDE
\uput[dl](0.5,0.5){A}\uput[dr](2.5,0.5){B}\uput[ur](2.5,2.5){C}\uput[u](1.5,3.5){D}\uput[ul](0.5,2.5){E}
\psline(2.5,2.5)(0.5,2.5)
\end{pspicture}}

\bigskip

\textbf{Partie 1}

\medskip

\begin{enumerate}
\item Donner, sans justification, les mesures des angles $\widehat{\text{DEC}}$  et $\widehat{\text{DCE}}$.
\item Montrer que le côté [DE] mesure environ $3,5$~cm au dixième de centimètre près.
\item Calculer l'aire du motif initial. Donner une valeur approchée au centimètre carré près.
\end{enumerate}

\bigskip

\textbf{Partie 2}

\medskip

\parbox{0.4\linewidth}{On réalise un pavage du plan en partant du motif initial et en utilisant différentes transformations du plan.

Dans chacun des quatre cas suivants, donner sans justifier une transformation du plan qui permet de passer :

\begin{enumerate}
\item Du motif 1 au motif 2
\item Du motif 1 au motif 3
\item Du motif 1 au motif 4
\item Du motif 2 au motif 3
\end{enumerate}}
\hfill
\parbox{0.57\linewidth}{
\def\motifa{
\psset{unit=1cm}
\begin{pspicture}(2.1,3.5)
\psline[fillstyle=solid,fillcolor=lightgray](0,0)(2.1,0)(2.1,2.1)(1.05,3.22)(0,2.1)(0,0)
\end{pspicture}
}%ABCDE
\def\motifb
{\psset{unit=1cm}
\begin{pspicture}(2.1,3.5)
\psline(0,0)(2.1,0)(2.1,2.1)(1.05,3.22)(0,2.1)(0,0)
\end{pspicture}
}
\psset{unit=0.75cm}
\begin{pspicture}(11,10)
\psline(0,0)(4,0)(5,1)(6,0)(8,0)(8,2)(6,2)(5,1)(4,2)(2,2)(2,0)
\psline(8,2)(10,2)(11,3)(10,4)(8,4)(10,4)(10,6)(8,6)(7,5)(8,4)(8,2)
\psline(8,6)(8,8)(6,8)(4,8)(4,10)(2,10)(2,8)(3,7)(4,8)
\psline(6,8)(6,6)(7,5)(6,4)(6,2)
\psline(6,6)(4,6)(3,7)(2,6)(0,6)(0,8)(2,8)
\psline(0,0)(0,2)(1,3)(2,2)(4,2)(4,4)(6,4)
\psline(1,3)(2,4)(4,4)(2,4)(2,6)
\psline(0,6)(0,4)(1,3)
\pspolygon[linewidth=2pt,fillstyle=solid,fillcolor=lightgray](2,4)(2,6)(3,7)(4,6)(4,4)
\pspolygon[linewidth=2pt,fillstyle=solid,fillcolor=lightgray](4,6)(4,4)(6,4)(7,5)(6,6)
\pspolygon[linewidth=2pt,fillstyle=solid,fillcolor=lightgray](4,2)(4,4)(6,4)(6,2)(5,1)
\pspolygon[linewidth=2pt,fillstyle=solid,fillcolor=lightgray](6,2)(6,4)(7,5)(8,4)(8,2)
\uput[l](2,4){A} \uput[dl](4,4){B} \uput[ur](4,6){C} \uput[u](3,7){D} 
\uput[ul](2,6){E} \uput[ur](6,6){F} \uput[dr](6,4){G} \uput[u](7,5){H} 
\uput[ur](8,4){I} \uput[dr](8,2){J} \uput[dr](6,2){K} \uput[d](5,1){L} 
\uput[dl](4,2){M}
\rput(3,5){motif 1} \rput(5,5){motif 2}\rput(7,3){motif 3}\rput(5,3){motif 4}
\end{pspicture}
}

\bigskip

\textbf{Partie 3}

\medskip

Suite à un agrandissement de rapport $\dfrac{3}{2}$ de la taille du motif initial, on obtient un motif agrandi.

\medskip

\begin{enumerate}
\item Construire en vraie grandeur le motif agrandi.
\item Par quel coefficient doit-on multiplier l'aire du motif initial pour obtenir l'aire du motif agrandi?
\end{enumerate}

\vspace{0.5cm}

\textbf{Exercice 4 \hfill 16 points}

\medskip

Jean possède 365 albums de bandes dessinées.Afin de trier les albums de sa collection, il les range par série et classe les séries en trois catégories: franco-belges, comics et mangas comme ci-dessous.

\begin{center}
\begin{tabularx}{\linewidth}{|*{3}{X|}}\hline
Séries franco-belges&Séries de comics&Séries de mangas\\ \hline
23 albums \og Astérix \fg&35 albums \og Batman \fg&85 albums \og One-Pièce \fg\\
22 albums \og Tintin\fg&90 albums \og Spider-Man \fg&65 albums \og Naruto \fg\\
45 albums \og Lucky-Luke \fg&&\\ \hline
\end{tabularx}
\end{center}

\medskip
 
Il choisit au hasard un album parmi tous ceux de sa collection.
\medskip

\begin{enumerate}
\item 
	\begin{enumerate}
		\item Quelle est la probabilité que l'album choisi soit un album \og Lucky-Luke\fg ?
		\item Quelle est la probabilité que l'album choisi soit un comics ?
		\item Quelle est la probabilité que l'album choisi ne soit pas un manga ?
	\end{enumerate}
\item Tous les albums de chaque série sont numérotés dans l'ordre de sortie en librairie et chacune des séries est complète du numéro 1 au dernier numéro.
	\begin{enumerate}
		\item Quelle est la probabilité que l'album choisi porte le numéro 1 ?
		\item Quelle est la probabilité que l'album choisi porte le numéro 40 ?
	\end{enumerate}
\end{enumerate}
\vspace{0.5cm}

\textbf{Exercice 5 \hfill 21 points}

\medskip

On considère les fonctions $f$ et $g$ suivantes: 

\[f :\: t \longmapsto  4t + 3\quad \text{et}\quad  g :\: t \longmapsto 6t.\]

Leurs représentations graphiques $\left(d_1\right)$ et $\left(d_2\right)$ sont tracées ci-dessous.

\begin{center}
\psset{xunit=2.25cm,yunit=0.4cm,comma=true}
\begin{pspicture}(-1,-3)(4.1,24)
\multido{\n=-1.0+0.1}{52}{\psline[linewidth=0.1pt](\n,-3)(\n,24)}
\multido{\n=-1.0+0.5}{11}{\psline[linewidth=0.7pt](\n,-3)(\n,24)}
\multido{\n=-3+1}{28}{\psline[linewidth=0.1pt](-1,\n)(4.1,\n)}
\multido{\n=0+5}{5}{\psline[linewidth=0.7pt](-1,\n)(4.1,\n)}
\psaxes[linewidth=1.25pt,Dy=5,Dx=0.5]{->}(0,0)(-1,-3)(4.1,24)
\psaxes[linewidth=1.25pt,Dy=5,Dx=0.5](0,0)(-1,-3)(4.1,24)
\psplot[plotpoints=2000,linecolor=red,linewidth=1.25pt]{-0.5}{4}{6 x mul}
\psplot[plotpoints=2000,linecolor=blue,linewidth=1.25pt]{-1}{4}{4 x mul 3 add}
\uput[d](3.5,17){\blue $\left(d_2\right)$}\uput[u](3,18){\red $\left(d_1\right)$}
\uput[dr](0,0){O}
\end{pspicture}
\end{center}

\medskip

\begin{enumerate}
\item Associer chaque droite à la fonction qu'elle représente.
\item Résoudre par la méthode de votre choix l'équation $f(t) = g(t)$.
\end{enumerate}

Camille et Claude décident de faire exactement la même randonnée mais Camille part $45$~min avant Claude. On sait que Camille marche à la vitesse constante de $4$ km/h et Claude marche à la vitesse constante de $6$~km/h.

\begin{enumerate}[resume]
\item Au moment du départ de Claude, quelle est la distance déjà parcourue par Camille ?
\end{enumerate}

On note $t$ le temps écoulé, exprimé en heure, depuis le départ de Claude. Ainsi $t = 0$ correspond au moment du départ de Claude.

\begin{enumerate}[resume]
\item Expliquer pourquoi la distance en kilomètre parcourue par Camille en fonction de $t$ peut s'écrire $4t + 3$.
\item Déterminer le temps que mettra Claude pour rattraper Camille.
\end{enumerate}
\end{document}