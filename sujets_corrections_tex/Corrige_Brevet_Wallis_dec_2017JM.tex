\documentclass[10pt]{article}
\usepackage[T1]{fontenc}
\usepackage[utf8]{inputenc}
\usepackage{fourier}
\usepackage[scaled=0.875]{helvet}
\renewcommand{\ttdefault}{lmtt}
\usepackage{amsmath,amssymb,makeidx}
\usepackage{fancybox}
\usepackage{tikz}
\usetikzlibrary{shapes.geometric}
\usepackage{mathtools,amsfonts,amssymb} % nécessaire pour taper des maths
\usepackage{pstricks}  % permet de faire des dessins
\usepackage{pstricks-add} % avec des possibilités élargies
\usepackage{pst-eucl}  		% permet de faire des dessins de géométrie simplement
\usepackage{tikz,tkz-tab} 	% permet de faire des tableaux de variations facilement
\usepackage{pst-text}  		%permet d'écrire du texte dans une zone math
\usepackage{enumerate}  	%Permet de renuméroter différement les questions
\usepackage{esvect}  		% permet de faire des beaux vecteurs
\usepackage{multicol}  		% permet de fusionner des colonnes
\usepackage{multirow}   	% permet de fusionner des lignes
\usepackage{colortbl}  		%couleur dans les tableaux
\usepackage{tabularx}  		% permet des tableaux plus compliqués mais pas simple à  utiliser.
\usepackage{graphicx} 		% pour l'insersion d'images, incompatible avec xcolor
\everymath{\displaystyle} %pour que les maths soient à  la bonne taille (limite, intégrale)
\usepackage{scratch}
\usepackage{marvosym}   %c'est pour le symbole euro : code \EUR{}
\usepackage[left=3.5cm, right=3.5cm, top=3cm, bottom=3cm]{geometry}
\newcommand{\vect}[1]{\overrightarrow{\,\mathstrut#1\,}}
%%%% Ajouts 2017
\newcommand{\textding}[1]{\text{\ding{#1}}}
\usepackage{pgfplots}
\usepgflibrary{shapes.callouts} % Pour des bulles de BD \begin{tikzpicture}[remember picture]
%\node[ellipse callout, draw] (hallo) {texte}; \end{tikzpicture}
\usepgflibrary{shapes.symbols}  % Pour des bulles de BD \begin{tikzpicture}[remember picture]
%\node[starburst,starburst points height=1cm, draw,blue] {texte}; \end{tikzpicture}
% p77,visualTikZ-fr17.pdf
\usepackage{animate}
\usetikzlibrary{mindmap}
\usetikzlibrary[mindmap]
% Puces
\newenvironment{Puces}[1][1cm]%
{\begin{list}%
	{$\bullet$}%
	{	\setlength{\leftmargin}{#1}% marge à gauche, par défaut=1cm
		\setlength{\itemsep}{3ptplus3ptminus2pt}% espacement entre item
		\setlength{\topsep}{3ptplus3ptminus2pt}% espacement entre le paragraphe précédent et le 1er item
	}}%
{\end{list}}
%\usepackage{tabularx,graphicx}
%\usepackage{multirow}
\usepackage[normalem]{ulem}
\usepackage{pifont}
\usepackage{textcomp} 
\newcommand{\euro}{\eurologo{}}
%Tapuscrit : Joëlle Meillier
\usepackage{pstricks,pst-plot,pst-tree,pstricks-add}
\newcommand{\R}{\mathbb{R}}
\newcommand{\N}{\mathbb{N}}
\newcommand{\D}{\mathbb{D}}
\newcommand{\Z}{\mathbb{Z}}
\newcommand{\Q}{\mathbb{Q}}
\newcommand{\C}{\mathbb{C}}
\renewcommand{\theenumi}{\textbf{\arabic{enumi}}}
\renewcommand{\labelenumi}{\textbf{\theenumi.}}
\renewcommand{\theenumii}{\textbf{\alph{enumii}}}
\renewcommand{\labelenumii}{\textbf{\theenumii.}}
\def\Oij{$\left(\text{O},~\vect{\imath},~\vect{\jmath}\right)$}
\def\Oijk{$\left(\text{O},~\vect{\imath},~\vect{\jmath},~\vect{k}\right)$}
\def\Ouv{$\left(\text{O},~\vect{u},~\vect{v}\right)$}
\usepackage{fancyhdr}
\usepackage[dvips]{hyperref}
\hypersetup{%
pdfauthor = {APMEP},
pdfsubject = {Brevet des collèges},
pdftitle = {Wallis et Futuna 2 décembre 2017},
allbordercolors = white,
pdfstartview=FitH} 
\usepackage[np]{numprint}
\usepackage[frenchb]{babel}
\renewcommand{\baselinestretch}{1.2}
%% Mes commandes 
% Met entre guillemets français
\def\guill#1{\og{}#1\fg{}}
\usepackage{cancel}% pour barrer des termes dans les formules
\input{xlop} % JM pour les opérations
%%% Table des nombres premiers  %%%%
\newcount\primeindex
\newcount\tryindex
\newif\ifprime
\newif\ifagain
\newcommand\getprime[1]{%
\opcopy{2}{P0}%
\opcopy{3}{P1}%
\opcopy{5}{try}
\primeindex=2
\loop
\ifnum\primeindex<#1\relax
\testprimality
\ifprime
\opcopy{try}{P\the\primeindex}%
\advance\primeindex by1
\fi
\opadd*{try}{2}{try}%
\ifnum\primeindex<#1\relax
\testprimality
\ifprime
\opcopy{try}{P\the\primeindex}%
\advance\primeindex by1
\fi
\opadd*{try}{4}{try}%
\fi
\repeat
}
\newcommand\testprimality{%
\begingroup
\againtrue
\global\primetrue
\tryindex=0
\loop
\opidiv*{try}{P\the\tryindex}{q}{r}%
\opcmp{r}{0}%
\ifopeq \global\primefalse \againfalse \fi
\opcmp{q}{P\the\tryindex}%
\ifoplt \againfalse \fi
\advance\tryindex by1
\ifagain
\repeat
\endgroup
}
%%% Décomposition en nombres premiers %%%

\newcommand\primedecomp[2][nil]{%
\begingroup
\opset{#1}%
\opcopy{#2}{NbtoDecompose}%
\opabs{NbtoDecompose}{NbtoDecompose}%
\opinteger{NbtoDecompose}{NbtoDecompose}%
\opcmp{NbtoDecompose}{0}%
\ifopeq
Je refuse de décomposer zéro.
\else
\setbox1=\hbox{\opdisplay{operandstyle.1}%
{NbtoDecompose}}%
{\setbox2=\box2{}}%
\count255=1
\primeindex=0
\loop
\opcmp{NbtoDecompose}{1}\ifopneq
\opidiv*{NbtoDecompose}{P\the\primeindex}{q}{r}%
\opcmp{0}{r}\ifopeq
\ifvoid2
\setbox2=\hbox{%
\opdisplay{intermediarystyle.\the\count255}%
{P\the\primeindex}}%
\else
\setbox2=\vtop{%
\hbox{\box2}
\hbox{%
\opdisplay{intermediarystyle.\the\count255}%
{P\the\primeindex}}}
\fi
\opcopy{q}{NbtoDecompose}%
\advance\count255 by1
\setbox1=\vtop{%
\hbox{\box1}
\hbox{%
\opdisplay{operandstyle.\the\count255}%
{NbtoDecompose}}
}%
\else
\advance\primeindex by1
\fi
\repeat
\hbox{\box1
\kern0.5\opcolumnwidth
\opvline(0,0.75){\the\count255.25}
\kern0.5\opcolumnwidth
\box2}%
\fi
\endgroup
}
\begin{document}
\setlength\parindent{0mm}
\pagestyle{fancy}
\thispagestyle{empty}
\rhead{\textbf{A. P{}. M. E. P{}.}}
\lhead{\small Corrigé du brevet des collèges}
\lfoot{\small{Wallis et Futuna}}
\rfoot{\small{2 décembre 2017}}
\begin{center}


{\Large \textbf{\decofourleft~Corrigé du brevet des collèges Wallis et Futuna~\decofourright\\[4pt]2 décembre 2017}}

\medskip

\textbf{THÉMATIQUE COMMUNE DE L'ÉPREUVE DE MATHÉMATIQUES-SCIENCES : LA SANTÉ}

\end{center}

\vspace{0.2cm}

\textbf{Exercice 1 :  \hfill 5 points}

\medskip

\textbf{Question 1} : Dans un club sportif, $\dfrac{1}{8}$ des adhérents ont plus de 42~ans et $\dfrac{1}{4}$, soit $\dfrac{2}{8}$ ont moins de 25~ans.

$\dfrac{1}{8}+\dfrac{2}{8}=\dfrac{3}{8}$. Il reste une proportion de $1 - \dfrac{3}{8} = \dfrac{8 - 3}{8} = \dfrac{5}{8}$ d'adhérents ayant un âge de 25 à  42~ans. \textbf{Réponse~C}.

\textbf{Question 2} : Pour augmenter le prix de 20~\%\, on multiplie le prix de départ par 1,20. $46~000\times 1,20=\np{55200}$. \textbf{Réponse~B}.

\textbf{Question 3} : Si toutes les longueurs sont multipliées par $k$, alors les aires sont multipliées par $k^2$ et les volumes sont multipliés par $k^3$. Ici, toutes les longueurs du cube sont multipliées par 3, donc le volume du cube est multiplié par $3^3$, soit par 27.  \textbf{Réponse~D}. \\[2mm]
\textbf{Question 4} : Les nombres 23 et 37 sont impairs, donc on élimine la réponse~D.

Les nombres 23 et 37 ne sont pas divisibles par 3 (on ne les trouve pas dans la table de multiplication du 3 ; ou la somme de leurs chiffres n’est pas divisible par 3 ($2+3=5$ et 5 n’est pas divisible par 3 ; $3+7 = 10$ et 10 n’est pas divisible par 3)), donc on élimine la réponse~B.

Tous les nombres entiers sont divisibles par 1, donc les nombres 23 et 37 ont 1 comme diviseur commun, donc on élimine la réponse~C.

Il ne reste que la bonne réponse. Les nombres 23 et 37 ont exactement deux diviseurs (1 et le nombre lui-même), donc ils sont premiers. \textbf{Réponse~A}.

\textbf{Question 5} : On calcule $f(3)$ (en remplaçant $x$ par 3). 

$3^2 - 2\times 3+7 = 9 - 6 + 7 =  10$. \quad  \textbf{Réponse~A}. 

\vspace{0.5cm}

\textbf{Exercice 2 :  \hfill 4 points}

\vspace{0.2cm}
Voici les tailles, en cm, de 29 jeunes plants de blé 10 jours après la mise en germination.

\vspace{0.2cm}

\begin{tabularx}{\linewidth}{|l|*{9}{>{\centering \arraybackslash}X|}}\hline
Taille (en cm) &0 &10 &15 &17 &18 &19 &20 &21 &22\\ \hline
Effectif &1 &4 &6 &2 &3 &3 &4 &4 &2\\ \hline
Effectif cumulé croissant&1 &5 &11 &13 &16 &19 &23 &27 &29\\ \hline
\end{tabularx}

\vspace{0.2cm}

\begin{enumerate}
\item $\dfrac{1\times 0 + 4\times10  + 6\times15  + 2\times17  + 3\times18  + 3\times 19 + 4\times20  +4\times21  +2\times 22}{29}=\dfrac{483}{29} \approx16,7$

La taille moyenne d'un jeune plant de blé est \textcolor{red}{\textbf{d’environ}} 16,7~cm 10 jours après la mise en germination.
\item 
	\begin{enumerate}
		\item L’effectif total est égal à 29. $29\div2=14,5$. La médiane est la 15\ieme\ donnée de la série de données ordonnée dans l’ordre croissant. La médiane de cette série est égale à 18~cm.
		\item Dire que la médiane de cette série est égale à 18~cm signifie qu’au moins la moitié des plants de blé mesurent 18~cm ou moins de 18~cm, 10 jours après la mise en germination.
	\end{enumerate}
\end{enumerate}

\newpage

\textbf{Exercice 3 :  \hfill 6 points}

\medskip

\begin{minipage}{9cm}
Pour gagner le gros lot à  une kermesse, il faut d'abord tirer une boule rouge dans une urne, puis obtenir un multiple de 3 en tournant une roue de loterie numérotée de 1 à  6.

L'urne contient 3 boules vertes, 2 boules bleues et 3 boules rouges.

\begin{enumerate}
\item Sur la roue de loterie, il y a deux issues (3 et 6) sur 6 issues qui réalisent l’évènement \guill{obtenir un multiple de $3$}.

La probabilité d'obtenir un multiple de $3$ est donc égale à $\dfrac{2}{6}$ $\left( \text{ou~} \dfrac{1}{3}\right)$
\item Dans l’urne, la probabilité de tirer une boule rouge est égale à $\dfrac{3}{8}$.

la probabilité de tirer une boule rouge dans une urne, puis d’obtenir un multiple de 3 sur la roue de loterie est égale à $\dfrac{3}{8}\times\dfrac{1}{3}$, soit $\dfrac{1}{8}$.

La probabilité qu'un participant gagne le gros lot est égale à $\dfrac{1}{8}$.
\end{enumerate}
\end{minipage}
\hspace{0.5cm}\begin{minipage}{5cm}
\psset{unit=0.85cm}
\begin{pspicture}(-2.5,-2.5)(2.8,2.5)
\pscircle(0,0){2.5}
\multido{\n=0+60,\na=30+60,\nb=1+1}{6}{\psline(2.5;\n)\rput(1.5;\na){\nb}}
\rput{-35}(2.6;-35){$\blacktriangleleft$}
\end{pspicture}
\end{minipage}

\begin{enumerate}
\item[\textbf{3.}] Comme on ne change pas le nombre de boules vertes et de boules bleues, il y a 5 boules vertes ou bleues.

Il faut que la moitié des boules soient rouges, donc il faut mettre en tout 5 boules rouges dans l'urne pour que la probabilité de tirer une boule rouge soit de $0,5$.
\end{enumerate}

\newpage

\textbf{Exercice 4 :  \hfill 5 points}

\medskip

\begin{enumerate}
\item On souhaite tracer le motif ci-dessous en forme de losange.

\vspace{0.2cm}

\begin{tabularx}{\linewidth}{|X|X|}\hline
Le motif \textbf{Losange} &Le bloc \textbf{Losange}\\
\psset{unit=0.6cm}
\begin{pspicture}(0,-5)(10,4)
\pspolygon(0.5,1)(5.6,1)(10,3.5)(4.9,3.5)
\psarc(0.5,1){4mm}{0}{30} \psarc(4.9,3.5){4mm}{210}{360}
\rput(1,2.5){\scriptsize Point de départ}
\psline{->}(0.5,2.2)(0.5,1.1)
\rput(1.6,1.3){\footnotesize 30\degres}\rput(5.1,2.5){\footnotesize 150\degres}
\psline{<->}(0.5,0.8)(5.6,0.8)
\uput[d](3.05,0.8){\footnotesize 60}
\end{pspicture}&\footnotesize{\begin{scratch}
\initmoreblocks{définir \namemoreblocks{Losange}}
\blockpen{stylo en position d'écriture}
\blockmove{avancer de \ovalnum{}}
\blockmove{tourner \turnleft{} de \ovalnum{30} degrés}
\blockmove{avancer de \ovalnum{}}
\blockmove{tourner \turnleft{} de \ovalnum{150} degrés}
\blockmove{avancer de \ovalnum{}}
\blockmove{tourner \turnleft{} de \ovalnum{} degrés}
\blockmove{avancer de \ovalnum{}}
\blockmove{tourner \turnleft{} de \ovalnum{} degrés}
\blockpen{relever le stylo}
\end{scratch}}\\ \hline
\end{tabularx}

\item On souhaite réaliser la figure ci-dessous construite à  partir du bloc \textbf{Losange} complété à  la question 1.

\parbox{0.6\linewidth}{
La figure :
\begin{center}
\psset{unit=0.75cm}
\begin{pspicture}(-2.5,-2.5)(2.5,2.5)
\def\losange1{\pspolygon(0,0)(1.275,0)(2.375,0.625)(1.1,0.625)}
\multido{\n=0+30}{12}{\rput{\n}(0;0){\losange1}}
\end{pspicture}
\end{center}

L’instruction \raisebox{-2.3ex}{\begin{scratch}\blockmove{s’orienter à  \ovalnum{90\selectarrownum}}\end{scratch}} signifie que l'on se dirige vers la droite. \\[2mm]
Les deux instructions à  placer dans la boucle pour
finir le script sont les \linebreak instruction \ding{'302}, puis \ding{'300}.}\hfill
\parbox{0.37\linewidth}{Le script : \\
\begin{scratch}
\blockinit{Quand \greenflag est cliqué}
\blockpen{effacer tout}
\blockmove{aller à  x: \ovalnum0 y: \ovalnum0}
\blockmove{s'orienter à  \ovalnum{90\selectarrownum}}
\blockrepeat{répéter \ovalnum{12} fois}
{\initmoreblocks{Losange}
\blockmove{tourner \turnleft{} de \ovalnum{30} degrés}
}
\end{scratch}}
\end{enumerate}

\vspace{0.5cm}

\textbf{Exercice 5 :  \hfill 9 points}

\medskip

\begin{enumerate}
\item 
	\begin{enumerate}
		\item Avec la formule $f(x) = 220 - x$, on remplace $x$ par 5.
		
$220-5 = 215$. La fréquence cardiaque maximale recommandée pour un enfant de $5$ ans est de 215~pulsations/minute.
		\item Avec la formule $g(x) = 208 - 0,7x$, on remplace $x$ par 5. 
		
$208- 0,7\times 5 = 208 - 3,5 = 204,5$. La fréquence cardiaque maximale recommandée pour un enfant de $5$ ans est de 204~pulsations/minute (on ne compte pas de demi-pulsation !).
	\end{enumerate}
\item  
	\begin{enumerate}
		\item Sur l'annexe 2, on complète le tableau de valeurs comme ci-dessous : \\[2mm]
		\begin{tabularx}{\linewidth}{|*{12}{>{\centering \arraybackslash}X|}}\hline
$x$		&5		&10		&20		&30		&40	&50		&60		&70		&80		&90		&100\\ \hline
$f(x)$	&215 	&210 	&200 	&190 	&180&170 	&160 	&150 	&140	&130 	&120 \\ \hline
$g(x)$	&204,5 	&201 	&194 	&187 	&180&173 	&166 	&159 	&152 	&145 	&138\\ \hline
\end{tabularx}
		\item Sur l'annexe 2, on a tracé en rouge la droite $d$ représentant la fonction $f$ dans le repère tracé.
		\item Sur le même repère, on a tracé en violet la droite $d'$ représentant la fonction $g$.
	\end{enumerate}
\item  Selon la nouvelle formule, à  partir de 40~ans la fréquence cardiaque maximale
recommandée est supérieure ou égale à  celle calculée avec l'ancienne formule. Ceci se voit dans le tableau : avant la colonne correspondant à 40~ans, $f(x)$ est supérieur à $g(x)$ et après cette colonne,  $f(x)$ est inférieur à $g(x)$.

Ceci se voit aussi sur la représentation graphique : avant le point d’intersection de $d$ et $d’$ correspondant à 40~ans, $d$ est au-dessus de $d’$ et après ce point,  $d$ est en-dessous de $d’$.
\item  L’exercice physique, pour une personne de $30$~ans, est le plus efficace lorsque la
fréquence cardiaque atteint 80\,\% de  187~pulsations/minute.

$\dfrac{80}{100}\times187=149,6$

Pour que l'exercice physique soit le plus efficace pour une personne de $30$~ans, la fréquence cardiaque doit être de 149 pulsations/minute (on ne compte pas 6 dixièmes de pulsation !).
\end{enumerate}

\vspace{0,5cm}

\textbf{Exercice 6 :  \hfill 7 points}

\medskip

Dans un laboratoire A, pour tester le vaccin contre la grippe de la saison hivernale prochaine, on a injecté la même souche de virus à  5 groupes comportant 29 souris chacun.

3 de ces groupes avaient été préalablement vaccinés contre ce virus.

Quelques jours plus tard, on remarque que :

\setlength\parindent{10mm}
\begin{itemize}
\item[$\bullet~~$] dans les $3$ groupes de souris vaccinées, aucune souris n'est malade ;
\item[$\bullet~~$] dans chacun des groupes de souris non vaccinées, $23$ souris ont développé la maladie.
\end{itemize}
\setlength\parindent{0mm} 

\medskip
 
\begin{enumerate}
\item 
	\begin{enumerate}
		\item Il y a $5$~groupes de $29$~souris. $5\times29=145$.
		
Il y a $2$~groupes de souris non vaccinées contenant chacun $23$~souris ayant développé la maladie.$2\times23=46$.
		
		 La proportion de souris malades lors de ce test est $\dfrac{46}{145}$ car il y a $46$~souris ayant développé la maladie sur $145$~souris.
		\item Les décompositions en facteurs premiers de 46 et 145 sont : \quad $46=2\times23$ et $145=5\times29$. \\
		Ces deux décompositions permettent de dire que le seul diviseur commun à 46 et 145 est 1, on ne peut donc pas simplifier cette fraction.
	\end{enumerate}	
\end{enumerate}
		
Dans un laboratoire B on informe que $\dfrac{140}{870}$ des souris ont été malades.

\begin{enumerate}		
\item[\textbf{2.}] 
	\begin{enumerate}
		\item  ~	
	\vspace{-0.5cm}
	\hspace{-0.3cm}\getprime{20}%
\primedecomp{140}

\vspace{0.4cm}

La décomposition en facteurs premiers de 140 est : \quad $140=2\times2\times5\times7$.

\hspace{-0.3cm}\getprime{20}%
\primedecomp{870}

\vspace{0.2cm}

La décomposition en facteurs premiers de 870 est : \quad $870 = 2 \times 3\times 5 \times 29$.
		\item $\dfrac{140}{870}=\dfrac{\cancel{2}\times2\times\cancel{5}\times7}{\cancel{2}\times3\times\cancel{5}\times29} = \dfrac{14}{87}$.
		
La forme irréductible de la proportion de souris malades dans le laboratoire B est $\dfrac{14}{87}$.
	\end{enumerate}
\end{enumerate}

\vspace{0,5cm}

\textbf{Exercice 7 :  \hfill 9 points}

\begin{minipage}{7.5cm}
\begin{enumerate}
\item Le triangle $ABC$ est rectangle en $A$, donc d’après le théorème de Pythagore :

$BC^2 = BA^2 + AC^2$

$BC^2 = 300^2 + 400^2$

$BC^2 = \np{90000} + \np{160000}$

$BC^2 = \np{250000}$ 

$BC=500$~m.
\end{enumerate}
\end{minipage}
\begin{minipage}{6cm}
\psset{unit=0.65cm,arrowsize=2pt 4}
\begin{pspicture}(10.5,5)
%\psgrid
\psline[ArrowInside=->](1.7,4)(0.5,2)(10.2,3.8)(8.5,0.5)
\psline[linestyle=dashed](1.7,4)(8.5,0.5)
\rput{-118}(1.7,4){\psframe(0.25,0.25)}
\rput{62}(8.5,0.5){\psframe(0.25,0.25)}
\uput[u](1.7,4){$A$ (départ)}\uput[dl](0.7,2){$B$}\uput[u](4.3,2.8){$C$}\uput[ur](10.2,3.8){$D$}
\uput[r](8.5,0.5){$E$ (arrivée)}
\rput(0.5,3.3){300~m}\rput(3.5,3.6){400~m}\rput(7.3,1.7){\np{1000}~m}
\end{pspicture}
\end{minipage}

\begin{enumerate}
\item[\textbf{2.}] Les triangles $ABC$ et $CDE$ ont deux angles de même mesure : l’angle droit et l’angle au sommet $C$, ils sont donc semblables.

Le triangle $CDE$ est un agrandissement du triangle $ABC$.

Si $k$ est le coefficient d’agrandissement, alors on a :

$\np{1000}=k\times 400$ \hspace{1cm} ; \hspace{1cm} $ED=k\times 300$ \hspace{1cm} et \hspace{1cm} $CD=k\times 500$

Avec la première égalité, on obtient $k=\dfrac{1~000}{400}$, soit $k=2,5$.

Avec la deuxième égalité, on obtient $ED=2,5\times 300$, soit $ED=750$~m. 
\item[\textbf{3.}] Avec la troisième égalité, on obtient $CD=2,5\times 500$, soit $CD=1~250$~m. 

$300+500+1~250+750=2~800$.

La longueur réelle du parcours $ABCDE$ est égale à \np{28000}~m.
\end{enumerate}

\newpage

\begin{center}
{\Large \textbf{Annexe 1 : exercice 4}}

\vspace{3cm}

\begin{scratch}
\initmoreblocks{définir \namemoreblocks{Losange}}
\blockpen{stylo en position d'écriture}
\blockmove{avancer de \ovalnum{60}}
\blockmove{tourner \turnleft{} de \ovalnum{30} degrés}
\blockmove{avancer de \ovalnum{60}}
\blockmove{tourner \turnleft{} de \ovalnum{150} degrés}
\blockmove{avancer de \ovalnum{60}}
\blockmove{tourner \turnleft{} de \ovalnum{30} degrés}
\blockmove{avancer de \ovalnum{60}}
\blockmove{tourner \turnleft{} de \ovalnum{150} degrés}
\blockpen{relever le stylo}
\end{scratch}

\end{center}
\newpage

\begin{center}
{\Large \textbf{Annexe 2 : exercice 5}}

\bigskip

\bigskip

\psset{xunit =0.08cm,yunit=0.08cm}
\begin{pspicture}(-5,-5)(100,220)
\multido{\n=0+5}{21}{\psline[linecolor=cyan,linewidth=0.3pt](\n,0)(\n,220)}
\multido{\n=0+5}{45}{\psline[linecolor=cyan,linewidth=0.3pt](0,\n)(100,\n)}
\psaxes[linewidth=1.25pt,Dx=10,Dy=10,labelFontSize=\scriptstyle]{->}(0,0)(0,0)(100,220)
\psaxes[linewidth=1.25pt,Dx=10,Dy=10,labelFontSize=\scriptstyle](0,0)(0,0)(100,220)
\psline[linewidth=1.25pt,linecolor=red](5,215)(100,120)
\psline[linewidth=1.25pt,linecolor=violet](5,204,5)(100,138)
\psline[linestyle=dashed](40,0)(40,180)
\psline[linestyle=dashed](0,180)(40,180)
\uput[r](90,120){\textcolor{red}{$d$}}
\uput[r](90,148){\textcolor{violet}{$d’$}}
\uput[r](0,217.5){fréquence cardiaque}
\uput[u](93,0){$x$ (âge)}
\end{pspicture}
\end{center}
\end{document}