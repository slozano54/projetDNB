\documentclass[10pt]{article}
\usepackage[T1]{fontenc}
\usepackage[utf8]{inputenc}
\usepackage{fourier}
\usepackage[scaled=0.875]{helvet}
\renewcommand{\ttdefault}{lmtt}
\usepackage{amsmath,amssymb,makeidx}
\usepackage[normalem]{ulem}
\usepackage{fancybox}
\usepackage{ulem}
\usepackage{dcolumn}
\usepackage{textcomp}
\usepackage{enumitem}
\usepackage{ifthen}
\usepackage{tabularx}
\usepackage{multirow}
\usepackage{colortbl}
\usepackage{graphicx}
\DeclareUnicodeCharacter{00A0}{~}
%Merci à :http://maths.ac-noumea.nc/spip.php?article620
\usepackage{lscape}
\usepackage{pst-fun} 
\newcommand{\euro}{\eurologo{}}
\usepackage{pstricks,pst-plot,pst-node,pstricks-add}
\usepackage[left=3cm, right=3cm, top=2cm, bottom=2cm]{geometry}
\newcommand{\R}{\mathbb{R}}
\newcommand{\N}{\mathbb{N}}
\newcommand{\D}{\mathbb{D}}
\newcommand{\Z}{\mathbb{Z}}
\newcommand{\Q}{\mathbb{Q}}
\newcommand{\C}{\mathbb{C}}
\usepackage{scratch}
\newcommand{\vect}[1]{\overrightarrow{\,\mathstrut#1\,}}
\renewcommand{\theenumi}{\textbf{\arabic{enumi}}}
\renewcommand{\labelenumi}{\textbf{\theenumi.}}
\renewcommand{\theenumii}{\textbf{\alph{enumii}}}
\renewcommand{\labelenumii}{\textbf{\theenumii.}}
\def\Oij{$\left(\text{O}~;~\vect{\imath},~\vect{\jmath}\right)$}
\def\Oijk{$\left(\text{O}~;~\vect{\imath},~\vect{\jmath},~\vect{k}\right)$}
\def\Ouv{$\left(\text{O}~;~\vect{u},~\vect{v}\right)$}
\usepackage{fancyhdr}
\usepackage{hyperref}
\hypersetup{%
pdfauthor = {APMEP},
pdfsubject = {Brevet des collèges},
pdftitle = {Nouvelle-Calédonie 14 décembre 2020},
allbordercolors = white,pdfstartview=FitH}   
\usepackage[french]{babel}
\usepackage[np]{numprint}
\begin{document}
\setlength\parindent{0mm}
\rhead{\textbf{A. P{}. M. E. P{}.}}
\lhead{\small Brevet des collèges}
\lfoot{\small{Nouvelle-Calédonie}}
\rfoot{\small{14 décembre 2020}}
\marginpar{\rotatebox{90}{\textbf{A. P{}. M. E. P{}.}}}
\pagestyle{fancy}
\thispagestyle{empty}

\begin{center}\textbf{Durée : 2 heures}

\vspace{0,5cm}

{\Large\textbf{\decofourleft~Diplôme national du Brevet
Nouvelle--Calédonie~\decofourright}}\\[5pt]
{\Large \textbf{14 décembre 2020}}

\medskip

ATTENTION : ANNEXE pages  et 8/8 est à rendre avec la copie

\medskip

\emph{L'usage de calculatrice avec mode examen activé est autorisé.\\
L'usage de calculatrice sans mémoire \og type collège \fg{} est autorisé}

\end{center}

\vspace{0,5cm}

\textbf{Exercice 1 : QCM \hfill 18 points}

\medskip

%\emph{Cet exercice est un questionnaire à choix multiples (QCM). Pour chaque question, une seule des trois réponses proposées est exacte.\\
%Sur la copie, indiquer le numéro de la question et la réponse {\rm A}, {\rm B} ou {\rm C} choisie.\\
%Aucune justification n'est demandée.\\
%Aucun point, ne sera enlevé en cas de mauvaise réponse.}
%
%\begin{center}
%\begin{tabularx}{\linewidth}{|c|m{4cm}|m{2.5cm}|*{3}{>{\centering \arraybackslash}X|}}\hline
%\multicolumn{3}{|c|}{\textbf{Propositions}}& \textbf{Réponse A}& \textbf{Réponse B}&\textbf{Réponse C}\\ \hline
%\textbf{1.}&\multicolumn{2}{|m{6.5cm}|}{$\dfrac{5}{3} - \dfrac{1}{3}\times \dfrac{3}{2}$ est égal à :}\rule[-4mm]{0mm}{8mm}&$\dfrac{2}{3}$&$2$&$\dfrac{7}{6}$\\ \hline
%\textbf{2.}&\multicolumn{2}{|m{6.5cm}|}{L'écriture scientifique de $245 \times 10^{-5}$ est :}\rule[-4mm]{0mm}{8mm}&$245 \times 5$&$2,45 \times 10^{-3}$&$2,45 \times 10^{-7}$\\ \hline
%\textbf{3.}&\multirow{2}{4cm}{On donne les durées en minutes entre les différents arrêts d'une ligne de bus :
%3~;~2~;~4~;~3~;~7~;~9~;~7.}\rule[-5mm]{0mm}{9mm}&La durée moyenne est:&3 min&4 min&5 min\\ \cline{1-1}\cline{3-6}
%\textbf{4.}&&La durée médiane est :&3 min&4 min&5 min\\ \hline
%\textbf{5.}&\multicolumn{2}{|m{6.5cm}|}{Un jeu de 32 cartes comporte 4 rois.
%
%On tire au hasard une carte du jeu.
%
%Quelle est la probabilité d'obtenir un roi ?}&$\dfrac{1}{8}$&$\dfrac{1}{32}$&$\dfrac{3}{32}$\\ \hline 
%\textbf{6.}&\multicolumn{2}{|m{6.5cm}|}{Une ville située sur l'équateur peut avoir pour coordonnées :}&(45\degre N~;~ 45\degre E)& (78\degre N~;~0\degre E)& (0\degre N~;~ 78\degre O)\\ \hline
%\end{tabularx}
%\end{center}

\begin{enumerate}
\item $\dfrac{5}{3} - \dfrac{1}{3}\times \dfrac{3}{2} = \dfrac{5}{3} - \dfrac{1}{2} = \dfrac{10}{6} - \dfrac{3}{6} = \dfrac{7}{6}$.
\item $245 \times 10^{-5} = 2,45 \times 10^2 \times 10^{-5} = 2,45 \times 10^{-3}$.
\item $\bullet~~$Durée moyenne : $\dfrac{3 + 2 + 4 + 3 + 7 + 9 + 7}{7} = \dfrac{35}{7} = 5$~(min).

\item $\bullet~~$Durée médiane : $2 < 3 \leqslant 3 < \textbf{4} < 7 \leqslant 7 < 9$, le temps médian est 4~(min).
\item On a $p(\text{Roi}) = \dfrac{4}{32} = \dfrac{1}{8}$.
\item (0\degre N~;~ 78\degre O) : latitude nulle.
\end{enumerate}

\vspace{0,5cm}

\textbf{Exercice 2 : La facture \hfill 8 points}

\medskip

%Un prix TTC (Toutes Taxes Comprises) s'obtient en ajoutant la taxe appelée TGC (Taxe Générale sur la Consommation) au prix HT (Hors Taxes).
%
%En Nouvelle-Calédonie, il existe quatre taux de TGC selon les cas : 22\,\%, 11\,\%, 6\,\% et 3\,\%.
%
%\smallskip
%
%Alexis vient de faire réparer sa voiture chez un carrossier.
%
%Voici un extrait de sa facture qui a été tâchée par de la peinture. 
%
%Les colonnes B, D et E désignent des prix en francs.

\begin{center}
\begin{tabularx}{\linewidth}{|c|l|*{4}{>{\centering \arraybackslash}X|}}\hline
&A&B&C&D&E\\ \hline
1& \textbf{Référence}	&Prix HT&TGC (en \,\%)&Montant TGC&Prix TTC\\ \hline
2& Phare avant			&\np{64000}	&22\,\%	&\np{14080}	&\np{78080} \\ \hline
3& Pare choc			&\np{18000}	&22\,\%	&\pscurve*(-0.5,-0.1)(0,-0.1)(0.6,0.12)(0.8,0.2)(0.5,0.28)(0,0.25)(-0.25,0.3)(-0.5,-0.1)			&\np{21960}\\ \hline
4& Peinture				&\np{11700}	&11\,\%	&\np{1287}	&\np{12987}\\ \hline 
5& Main d'œuvre			&\np{24000}	&\pscurve*(-0.5,0)(0,-0.1)(0.6,0.1)(0.8,0.18)(0.5,0.28)(0,0.25)(-0.25,0.3)(-0.5,0)		&\np{1440}	&\np{25440}\\ \hline
6&\multicolumn{2}{c}{~}&\multicolumn{2}{r|}{\textbf{TOTAL À RÉGLER (en Francs)}}&\textbf{\np{138467}}\\ \hline
\end{tabularx}
\end{center}

\begin{enumerate}
\item %Quel est le montant TGC pour le pare-chocs ?
Le montant TGC pour le pare-chocs est égal à la différence $\np{21960} - \np{18000} = \np{3960}$~(francs).

On peut aussi calculer $\np{18000} \times \dfrac{22}{100} = \np{3960}$~(francs).
\item %Quel est le pourcentage de la TGC qui s'applique à la main d'œuvre ?
On a $\frac{\np{1440}}{\np{24000}} \times 100 = \dfrac{\np{1440}}{24} = 6$~(\,\%)
\item %La facture a été faite à l'aide d'un tableur.

%Quelle formule a été saisie dans la cellule E6 pour obtenir le total à payer ?
Dans la case C6 on écrit : = SOMME(E2:E5)
\end{enumerate}

\vspace{0,5cm}

\textbf{Exercice 3 : Programmes de calcul \hfill 11 points}

\medskip

%On donne les deux programmes de calcul suivants :
%
%\begin{center}
%\begin{tabularx}{\linewidth}{|X|X|}\hline 
%\multicolumn{1}{|c|}{\textbf{Programme A}}&\multicolumn{1}{|c|}{\textbf{Programme B}}\\
%~&~\\
%$\bullet~~$Choisir un nombre								&$\bullet~~$Choisir un nombre\\
%$\bullet~~$Soustraire $5$ à ce nombre						&$\bullet~~$Mettre ce nombre au carré\\
%$\bullet~~$Multiplier le résultat par le nombre de départ	&$\bullet~~$Soustraire 4 au résultat\\ \hline
%\end{tabularx}
%\end{center}
%
%\medskip

\begin{enumerate}
\item %Alice choisit le nombre 4 et applique le programme A. 

%Montrer qu'elle obtiendra $- 4$.
Elle obtient : $4 \to - 1 \to - 4$. 
\item %Lucie choisit le nombre $- 3$ et applique le programme B. 

%Quel résultat va-t-elle obtenir ?
Lucie obtient $- 3 \to 9 \to 5$.
\end{enumerate}

%Tom souhaite trouver un nombre pour lequel des deux programmes de calculs donneront le même résultat.
%
%Il choisit $x$ comme nombre de départ pour les deux programmes.
\begin{enumerate}[resume]
\item %Montrer que le résultat du programme A peut s'écrira $x^2 - 5x$.
On a successivement avec le programme A :  : $x \to x - 5 \to x(x - 5)$.
\item %Exprimer en fonction de $x$ le résultat obtenu avec le programme B.
On a successivement avec le programme B :  : $x \to x^2 \to x^2 - 4$.
\item %Quel est le nombre que Tom cherche ?
On veut trouver $x$ tel que :

$x(x - 5) = x^2 - 4$ ou $x^2 - 5x = x^2 - 4$ ou encore $4 = 5x$, soit en multipliant chaque membre par $\dfrac{1}{5}$, \: $x = \dfrac{4}{5} = 0,8$.
\end{enumerate}

%\textbf{Toute trace de recherche même non aboutie sera prise ,en compte dans la notation.}

\vspace{0,5cm}

\textbf{EXERCICE 4 : La régate \hfill 16 points}

\medskip

Dans la figure suivante, on donne les distances en mètres :

AB $= 400$, AC $= 300$, BC $= 500$ et CD $= 700$.

\medskip
\begin{center}
\begin{tabularx}{\linewidth}{lX} 
\psset{unit=1cm}
\begin{pspicture}(7,5.2)
%\psgrid
\pspolygon(1.5,3.2)(0.5,0.5)(6.2,5.14)(5,1.9)%ABDE
\uput[u](1.5,3.2){A} \uput[l](0.5,0.5){B} \uput[u](3,2.6){C} \uput[r](6.2,5.14){D} \uput[r](5,1.9){E} 
\end{pspicture}&\vspace{-2.5cm}\begin{tabular}{|l|} \hline

Les droites (AE) et (BD) se coupent en C \\
~\\
Les droites (AB) et (DE)sont parallèles\\ \hline
\end{tabular}\\
\end{tabularx}

\end{center}

\begin{enumerate}
\item %Calculer la longueur DE.
Les droites (AB) et (DE) étant parallèles, on peut écrire d'après le théorème de Thalès : 

$\dfrac{\text{DE}}{\text{AB}} = \dfrac{\text{CD}}{\text{BC}}$ soit $\dfrac{\text{DE}}{400} = \dfrac{700}{500}$,d'où en multipliant par 400 : $\text{DE} = 400 \times \dfrac{700}{500} = 400 \times \dfrac{7}{5} = 560$~(m).
\item %Montrer que le triangle ABC est rectangle.
On a BC$^2 = 500^2 = \np{25000}$ et $\text{AB}^2 + \text{AC}^2 = 400^2 + 300^2 = \np{16000} + \np{9000} = \np{25000}$.

On a donc  $\text{AB}^2 + \text{AC}^2 = \text{BC}^2$ : d'après la réciproque du théorème de Pythagore, le triangle ABC est rectangle en A.
\item %Calculer la mesure de l'angle $\widehat{\text{ABC}}$. Arrondir au degré.
Par définition du cosinus d'un angle aigu, dans le triangle ABC rectangle en A :

$\cos \widehat{\text{ABC}} = \dfrac{\text{AB}}{\text{BC}} = \dfrac{400}{500} = \dfrac{4}{5} = 0,8$.

La calculatrice donne, en mode degré : $\widehat{\text{ABC}} \approx 36,8$,soit 37\degres au degré près.
\end{enumerate}

%Lors d'une course les concurrents doivent effectuer plusieurs tours du parcours représenté ci-dessus. Ils partent du point A, puis passent par les points B, C, D et E dans cet ordre puis de nouveau par le point C pour ensuite revenir au point A.

%\smallskip
%
%Maltéo, le vainqueur, a mis 1~h 48~min pour effectuer les $5$ tours du parcours. La distance parcourue pour faire un tour est \np{2880}~m.

\begin{enumerate}[resume]
\item %Calculer la distance totale parcourue pour effectuer les $5$~tours du parcours.
\emph{Remarque : non demandé} : 

Pour calculer la longueur d'un parcours, il reste à calculer CE.

Or les droites (AB) et (DE) étant parallèles, la droite (AC) perpendiculaire à (AB) est aussi perpendiculaire à (DE), donc le triangle CDE est rectangle en E.

D'après le théorème de Pythagore :

$\text{CE}^2 + \text{ED}^2 = \text{CD}^2$ ou $\text{CE}^2 + 560^2 = 700^2$,soit $\text{CE}^2 = 700^2 - 560^2 = (700+560)\times (700-560) = \np{1260} \times 140 = \np{176400} $.

D'où CE $= \sqrt{\np{176400}} = 420$~(m).

Longueur d'un parcours : AB + BC + CD + DE + EC + CA $= 400 + 500 + 700 + 560 + 420 + 300 = \np{2880}$.

Les 5 tours représentent donc une longueur de $5 \times \np{2880} = \np{14400}$~(m) ou 14,4~(km).
\item %Calculer la vitesse moyenne de Maltéo. Arrondir à l'unité.
1~h 48~min = 60 + 48 = 108~min.
La vitesse moyenne est égale au quotient de la distance parcourue par le le temps mis pour faire les 5 tours :

$v = \dfrac{\np{14400}}{108}= \dfrac{\np{1600}}{12}= \dfrac{400}{3} \approx 133,33$~(m/min) soit $\approx 60 \times 133,33 = \np{7999,8}$~(m/h), soit enfin à peu près 8~(km/h).
\end{enumerate}

\vspace{0,5cm}

\textbf{EXERCICE 5 : La corde \hfill 7 points}

\medskip

%Le triangle ABC rectangle en B ci-dessous est tel que AB $= 5$ m et AC $= 5,25$ m.


\medskip

\begin{enumerate}
\item Le triangle ABC étant rectangle en B, le théorème de Pythagore permet d'écrire :

$\text{AB}^2 + \text{BC}^2 = \text{AC}^2$, soit $5^2 + \text{BC}^2 = 5,25^2$ ou encore $\text{BC}^2 = 5,25^2 - 5^2 = \np{2,5625} \approx \np{1,60078}$ soit 1,6 m au dixième près.

%\parbox{0.47\linewidth}{Calculer, en m, la longueur BC.
%
% Arrondir au dixième.}\hfill \parbox{0.5\linewidth}{
%\psset{unit=1cm}
%\begin{pspicture}(4.5,2.4)
%\pspolygon(0.5,0.5)(4.2,0.5)(4.2,2.2)%ABC
%\uput[l](0.5,0.5){A} \uput[r](4.2,0.5){B} \uput[ur](4.2,2.2){C}
%\psframe(4.2,0.5)(4,0.7) 
%\end{pspicture}}
%\end{enumerate}

%Une corde non élastique de $10,5$ m de long est fixée au sol par ses deux extrémités entre deux poteaux distants de $10$~m.
\item Si la corde est tendue en son milieu on a la figure suivante composée de deux triangles rectangles identiques à celui de la question \textbf{1.}:

\begin{center}
\psset{unit=1cm}
\begin{pspicture}(10,2)
\pspolygon(0,0)(10,0)(5,1.6)
\psline{<->}(5,0)(5,1.55)\uput[l](5,0.7625){1,6}
\psline{<->}(5,1)(10,-0.2)\uput[d](7.5,-0.2){5}
\rput(5.36,0.77){\includegraphics[width=0.64cm]{Melvin}}
\end{pspicture}
\end{center}

%\parbox{0.47\linewidth}{Melvin qui mesure 1,55 m pourrait-il passer sous cette corde sans se baisser en la soulevant par le milieu ?}\hfill
%\parbox{0.5\linewidth}{
%\psset{unit=1cm,arrowsize=2pt 3}
%\begin{pspicture}(7.5,4.2)
%\psline{<->}(0.4,0.5)(7.4,0.5)
%\rput(3.5,2.5){\includegraphics[width=1.5cm]{Melvin}}
%\psline(0.4,1)(0.4,0.5)\psline(7.4,1)(7.4,0.5)
%\qdisk(0.4,1){1mm}\qdisk(7.4,1){1mm}
%\pscurve(0.4,1)(1,1.2)(2,1.9)(3.3,2.3)(4,2.1)(5,1.7)(6,1.4)(7.4,1)
%\uput[d](3.9,0.5){10~m}
%\end{pspicture}
%}
Comme $1,55 < 1,60$, Melvin qui mesure 1,55 m pourra passer sous cette corde sans se baisser en la soulevant par le milieu.
%\textbf{Toute trace de recherche même non aboutie sera prise en compte dans la notation.}
\end{enumerate}

\vspace{0,5cm}

\textbf{EXERCICE 6 : Les étiquettes \hfill 14 points}

\medskip
 
\begin{enumerate}
\item %Justifier que le nombre 102 est divisible par 3.
$\bullet~~$Comme $1 + 0 + 2 = 3$, 102 est un multiple de 3 (critère de divisibilité par 3 ;

$\bullet~~$$102 = 90 + 12 = 3\times 30 + 3 \times 4 = 3\times (30 + 4) = 3\times 34$.

102 est un multiple de 3  : il est divisible par 3.
\item On donne la décomposition en produits de facteurs premiers de 85 : $85 = 5 \times 17$.

%Décomposer 102 en produits de facteurs premiers.
On a vu que $102 = 3 \times 34 = 3 \times 2 \times 17 = 2 \times 3 \times 17$.
\item Donner 3 diviseurs non premiers du nombre 102.

$2 \times 3 = 6 \:;\:  2 \times 17 = 34 \:;\: 3 \times 17 = 51$ sont trois diviseurs de 102 non premiers. 
%\end{enumerate}

%Un libraire dispose d'une feuille cartonnée de 85 cm sur 102 cm.
%
%Il souhaite découper dans celle-ci, en utilisant toute la feuille, des étiquettes carrées. 
%
%Les côtés de ces étiquettes ont tous la même mesure.
%\begin{enumerate}[resume]
\item %Les étiquettes peuvent-elles avoir $34$ cm de côté ? Justifier. 
Si toute la feuille est utilisée c'est que la longueur et la largeur sont des multiples des côtés du carré. Ces côtés ont donc une  longueur $c$ qui divise à la fois 102 et 85.

Or 34 ne divise pas 85 (car 2 divise 34 mais ne divise pas 85). les étiquettes ne peuvent pas faire 34cm de côté.
\item %Le libraire découpe des étiquettes de $17$ cm de côté.

%Combien d'étiquettes pourra-t-il découper dans ce cas ?
Par contre 17 divise 85 ($85 = 5 \times 17$) et 17 divise 102 ($102 = 17 \times 6$).

Les étiquettes rentrent 5 fois en largeur et 6 fois en longueur : il y en aura donc $5 \times 6 = 30$ par feuille.

\emph{Remarque } : on peut aussi utiliser les aires.

Une étiquette a une aire de $17 \times 17 = 289$ et la feuille une aire de $85 \times 102 = \np{8670}$.

On pourra donc faire $\dfrac{\np{8670}}{289} = 30$ étiquettes dans une feuille.
\end{enumerate}

\vspace{0,5cm}

\textbf{EXERCICE 7 : L'habitation \hfill 15 points}

\medskip

%Nolan souhaite construire une habitation.
%
%Il hésite entre une \textbf{case} et une \textbf{maison} en forme de prisme droit.
%
%La case est représentée par un cylindre droit d'axe (OO$'$) surmontée d'un cône de révolution de sommet S.
%
%Les dimensions sont données sur les figures suivantes.
%
%$x$ \textbf{représente à la fois le diamètre de la case et la longueur AB du prisme droit.}
%
%\begin{center}
%\begin{tabularx}{\linewidth}{*{2}{>{\centering \arraybackslash}X}}
%\psset{unit=0.8cm,arrowsize=2pt 3}
%\definecolor{mongris}{gray}{0.6}
%\definecolor{grisleger}{gray}{0.8}
%\begin{pspicture}(-3,-0.5)(3,5)
%\pscustom[fillstyle=solid,fillcolor=grisleger]%haut
%{\psellipticarc(0,0)(3,0.45){180}{0}
%\psline(3,2.5)(0,3.75)(-3,2.5)
%}
%\pscustom[fillstyle=solid,fillcolor=mongris]
%{\psellipticarc[linestyle=dashed](0,2.5)(3,0.45){0}{180}
%\psellipticarc(0,2.5)(3,0.5){180}{360}
%}
%\pscustom[fillstyle=solid,fillcolor=grisleger]
%{\psellipticarc(0,2.5)(3,0.5){180}{360}
%\psline(3,2.5)(3,0)
%\psellipticarc[linestyle=dashed](0,0)(3,0.49){180}{0}
%\psline(-3,0)(-3,2.5)}
%\pscustom[fillstyle=solid,fillcolor=mongris]
%{\psellipticarc[linestyle=dashed](0,0)(3,0.45){0}{180}
%\psellipticarc(0,0)(3,0.5){180}{360}
%}
%\psdots(0,0)(0,2.5)(0,3.75)
%\uput[ur](0,0){O}\uput[ur](0,2.5){O$'$}\uput[r](0,3.75){S}
%\psline{<->}(-2.5,-0.272)(2.5,0.272)\uput[dr](0,0){$x$}
%\psline(0,0)(0,3.75)\rput{90}(-0.2,1.25){2 m}\rput{90}(-0.2,3.3){1 m}
%\end{pspicture}&
%\psset{unit=0.8cm}
%\begin{pspicture}(7.3,4.3)
%\psline(0.5,0.9)(3.5,0.5)(7,1)(7,2.5)(3.5,2)(3.5,0.5)
%\psline(7,2.5)(3.9,4.1)(0.5,3.6)(3.5,2)
%\psline(0.5,0.9)(0.5,3.6)
%\uput[d](3.5,0.5){A}\uput[d](7,1){B}\uput[d](5.25,0.75){$x$}
%\uput[d](2,0.6){5 m}\uput[l](0.3,1.8){2 m}\uput[l](0.3,3.125){1 m}
%\psline[linestyle=dotted,linewidth=1.25pt](3.5,2)(0.5,2.7)
%\psline[linestyle=dotted,linewidth=1.25pt](7,1)(4,1.4)(4,4.1)
%\psline[linestyle=dotted,linewidth=1.25pt](0.5,0.9)(4,1.4)
%\psline[linewidth=0.2pt]{<->}(0.3,0.9)(0.3,2.7)
%\psline[linewidth=0.2pt]{<->}(0.3,3.6)(0.3,2.7)
%\end{pspicture}
%\end{tabularx}
%\end{center}

\textbf{Partie 1 :}

\medskip

Dans cette partie, on considère que $x = 6$ m.

\medskip

\begin{enumerate}
\item %Montrer que le volume exact de la partie cylindrique de la case est $18\pi$ m$^3$.

Le diamètre a une longueur de 6~m. Donc avec $r = 3$, le volume du cylindre est égal à :

$\pi \times 3^2 \times 2 = 18\pi$~m$^3$.
\item %Calculer le volume de la partie conique. Arrondir à l'unité.
Le volume de la partie conique est égale à : 

$\dfrac{1}{3} \times \pi \times 3^2 \times 1 = 3\pi$~m$^3$, soit $\approx 9,42$ ou 9~m$^3$ à l'unité près.
\item %En déduire que le volume total de la case est environ $66$~m$^3$.
Le volume de la case est donc égal à : 

$18\pi + 3\pi = 21\pi \approx 65,97$, soit $\approx 66$~m$^3$ à l'unité près.
\end{enumerate}

\begin{center}
\begin{tabularx}{\linewidth}{|c *{2}{>{\centering \arraybackslash}X}|}\hline
\textbf{Rappels :}&
Cylindre rayon de base $r$ et de hauteur $h$&Cône
rayon de base $r$ et de hauteur $h$\\
&\psset{unit=1cm}
\begin{pspicture}(-1.25,-0.4)(1.25,2.3)
\psellipticarc(0,0)(1.25,0.4){180}{360}
\psellipse(0,1.8)(1.25,0.4)
\pscustom[fillstyle=solid,fillcolor=lightgray]{
\psellipticarc(0,1.8)(1.25,0.4){00}{180}
\psline(-1.25,1.8)(-1.25,0)
\psellipticarc(0,0)(1.25,0.4){180}{360}
\psline(1.25,0)(1.25,1.8)
}
\uput[r](1.25,0.9){$h$}\uput[d](0.55,1.9){$r$}
\psline(0,1.8)(1.15,1.9)
\end{pspicture}&\psset{unit=1cm}
\begin{pspicture}(1.5,-0.5)(1.5,3)
\psellipticarc(0,0)(1.5,0.4){180}{360}
\pscustom[fillstyle=solid,fillcolor=lightgray]
{
\psellipticarc(0,0)(1.5,0.4){180}{360}
\psline(1.5,0)(0,2.8)(-1.5,0)
}
\psline[linestyle=dashed,linewidth=1.25pt](0,2.8)(0,0)(1.5,0)
\uput[l](0,1.4){$h$}\uput[d](0.75,0){$r$}
\psellipticarc[linestyle=dashed,linewidth=1.25pt](0,0)(1.5,0.4){0}{180}
\end{pspicture}\\
&$\text{Volume} =\pi \times r^2 \times h$&$\text{Volume} =\dfrac{1}{3} \times \pi \times r^2 \times h$\\ \hline
\end{tabularx}
\end{center}

\textbf{Partie 2 :}

\medskip

\emph{Dans cette partie, le diamètre est exprimé en mètre, le volume en m$^3$.}

\medskip

Sur l'\textbf{annexe } page \pageref{annexe1}, on a représenté la fonction qui donne le volume total de la case en fonction de son diamètre $x$.

\medskip

\begin{enumerate}
\item %Par lecture graphique, donner une valeur approchée du volume d'une case de $7$ m de diamètre.

%Tracer des pointillés permettant la lecture.
On lit sur l'annexe $V(7) \approx 90$~m$^3$.
\end{enumerate}

%La fonction qui donne le volume de la maison en forme de prisme droit est définie par 

\[V(x) = 12,5 x.\]

\begin{enumerate}[resume]
\item %Calculer l'image de 8 par la fonction $V$.
On a $V(8) = 12,5 \times 8 = 100$~m$^3$.
\item %Quelle est la nature de la fonction $V$ ?
La fonction $V$ est une fonction linéaire.
\item %Sur l'\textbf{annexe } page \pageref{annexe1}, tracer la représentation graphique de la fonction $V$.
La représentation graphique de la fonction linéaire $V$ est une droite contenant l'origine.
\end{enumerate}

%Pour des raisons pratiques, la valeur maximale de $x$ est de $6$ m. Nolan souhaite choisir la construction qui lui offre le plus grand volume.
\begin{enumerate}[resume]
\item %Quelle construction devra-t-il choisir ? Justifier.
$\bullet~~$Le plus grand volume de la maison est donc $V(6) = 12,5 \times 6 = 75$~m$^3$.

$\bullet~~$Le plus grand volume de la case est donc $V(6) \approx 66$~m$^3$.

Nolan choisira donc la maison.
\end{enumerate}

\vspace{0,5cm}

\textbf{EXERCICE 8 : Scratch \hfill 11 points}

\medskip


Le script suivant permet de tracer le carré de côté $50$ unités .

\begin{center}
\begin{scratch}
\blockinit{quand \greenflag est cliqué}
\blockmove{s’orienter à \ovaloperator{\selectmenu{90}}}
\blockpen{stylo en position d’écriture}
\blockinfloop{répéter \ovalnum{4} fois}
{
\blockmove{avancer de \ovalnum{50}}
\blockmove{tourner \turnleft{} de \ovalnum{90} degrés}
}
\end{scratch}
\end{center}

\medskip

\begin{enumerate}
\item %Sur l'annexe page \pageref{annexe2}, compléter le script pour obtenir un triangle équilatéral de coté $80$ unités.

\medskip

On a lancé le script suivant :

\begin{center}
\begin{scratch}
\blockinit{quand \greenflag est cliqué}
\blockmove{s’orienter à \ovaloperator{\selectmenu{90}}}
\blockmove{mettre \selectmenu{longueur} à \ovalnum{40}}
\blockpen{stylo en position d’écriture}
\blockinfloop{répéter \ovalnum{12} fois}
{
\blockmove{avancer de \selectmenu{longueur}}
\blockmove{tourner \turnleft{} de \ovalnum{90} degrés}
\blockmove{ajouter  à \selectmenu{longueur}\ovalnum{10}}
}
\end{scratch}
\end{center}
\item %Entourer sur l'annexe page \pageref{annexe2}, la figure obtenue avec ce script.
Il suffit de compter le nombre de segments tracés : 12. Seule la figure 2 convient.
\end{enumerate}

\newpage
\begin{center}
\label{annexe1}
\textbf{\Large ANNEXE 1}

\medskip

\textbf{Exercice 7 :\\
Partie 2 :} question 1 et 3

\vspace{1cm}

\textbf{Volume de la case en fonction de} \boldmath $x$ \unboldmath

\bigskip

\psset{xunit=1.2cm,yunit=0.06cm,arrowsize=2pt 4}
\begin{pspicture}(11.4,220)
\multido{\n=0+1}{11}{\psline[linewidth=0.2pt,linecolor=lightgray](\n,0)(\n,220)}
\multido{\n=0+20}{10}{\psline[linewidth=0.2pt,linecolor=lightgray](0,\n)(11,\n)}
\multido{\n=0+4}{56}{\psline[linewidth=0.4pt,linecolor=lightgray](0,\n)(11,\n)}
\psaxes[linewidth=1.25pt,Dy=20]{->}(0,0)(0,0)(11,220)
\psaxes[linewidth=1.25pt,Dy=20](0,0)(0,0)(11,220)
\psplot[plotpoints=2000,linewidth=1.25pt,linecolor=red]{0}{11}{x dup mul 3.141592 mul 7 mul 12 div}
\psplot[plotpoints=2000,linewidth=1.25pt,linecolor=blue]{0}{6}{x 12.5 mul}
\uput[u](10.9,0){$x$}
\uput[r](0,218){$V(x) ~\left(\text{m}^3\right)$}
\psline[linestyle=dashed,linewidth=1.5pt, linecolor=red,ArrowInside=->]{->}(7,0)(7,89.8)(0,89.9)
\uput[l](0,89.8){\red $\approx 90$}
\end{pspicture}
\end{center}

\newpage
\begin{center}
\label{annexe2}
\textbf{\Large ANNEXE 2}

\vspace{2cm}

\textbf{Exercice 8 question 1}

\bigskip

\textbf{Script à compléter}

\bigskip

\begin{scratch}
\blockinit{quand \greenflag est cliqué}
\blockmove{s’orienter à \ovalnum{\selectmenu{90}}}
\blockpen{stylo en position d’écriture}
\blockinfloop{répéter \ovalnum{3} fois}
{
\blockmove{avancer de \ovalnum{80}}
\blockmove{tourner \turnleft{} de \ovalnum{120} degrés}
}
\end{scratch}


\vspace{1.5cm}

\textbf{Exercice 8 question 2}

\bigskip

\begin{tabularx}{\linewidth}{*{3}{>{\centering \arraybackslash}X}}
Figure 1&Figure 2& Figure 3\\
\psset{unit=1mm,linecolor=red}
\begin{pspicture}(33,33)
%\psgrid
\psline(0,0)(0,34)(32,34)(32,4)(4,4)(4,30)(28,30)(28,8)(8,8)(8,26)(24,26)
(24,12)(12,12)(12,22)(20,22)(20,16)(16,16)
\end{pspicture}&
\psset{unit=1mm,linecolor=red}
\begin{pspicture}(34,34)
%\psgrid
\psline(0,0)(0,34)(32,34)(32,4)(4,4)(4,30)(28,30)(28,8)(8,8)(8,26)(24,26)
(24,12)(12,12)
\end{pspicture}&
\psset{unit=1mm,linecolor=red}
\begin{pspicture}(-20,-12)(20,20)
%\psgrid
\psline(20;210)(18;90)(16.2;-30)(14.58;210)(13.122;90)(11.8098;-30)(10.629;210)(9.56;90)(8.61;-30)(7.75;210)(6.974;90)(6.276;-30)(5.649;210)
\end{pspicture}
\end{tabularx}
\end{center}
\end{document}