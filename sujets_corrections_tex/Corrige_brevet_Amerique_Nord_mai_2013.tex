\documentclass[10pt]{article}
\usepackage[T1]{fontenc}
\usepackage[utf8]{inputenc}%ATTENTION codage UTF8
\usepackage{fourier}
\usepackage[scaled=0.875]{helvet}
\renewcommand{\ttdefault}{lmtt}
\usepackage{amsmath,amssymb,makeidx}
\usepackage{fancybox}
\usepackage{tabularx}
\usepackage{multirow}
\usepackage[normalem]{ulem}
\usepackage{pifont}
\usepackage{textcomp} 
\newcommand{\euro}{\eurologo{}}
%Tapuscrit : Denis Vergès
\usepackage{pstricks,pst-plot,pst-tree,pstricks-add}
\newcommand{\R}{\mathbb{R}}
\newcommand{\N}{\mathbb{N}}
\newcommand{\D}{\mathbb{D}}
\newcommand{\Z}{\mathbb{Z}}
\newcommand{\Q}{\mathbb{Q}}
\newcommand{\C}{\mathbb{C}}
\usepackage[left=3.5cm, right=3.5cm, top=3cm, bottom=3cm]{geometry}
\newcommand{\vect}[1]{\overrightarrow{\,\mathstrut#1\,}}
\renewcommand{\theenumi}{\textbf{\arabic{enumi}}}
\renewcommand{\labelenumi}{\textbf{\theenumi.}}
\renewcommand{\theenumii}{\textbf{\alph{enumii}}}
\renewcommand{\labelenumii}{\textbf{\theenumii.}}
\def\Oij{$\left(\text{O},~\vect{\imath},~\vect{\jmath}\right)$}
\def\Oijk{$\left(\text{O},~\vect{\imath},~\vect{\jmath},~\vect{k}\right)$}
\def\Ouv{$\left(\text{O},~\vect{u},~\vect{v}\right)$}
\usepackage{fancyhdr}
\usepackage[dvips]{hyperref}
\hypersetup{%
pdfauthor = {APMEP},
pdfsubject = {Corrigé du brevet des collèges},
pdftitle = {Amérique du Nord  juin 2013},
allbordercolors = white}
\usepackage[np]{numprint}
\usepackage[frenchb]{babel}
\begin{document}
\setlength\parindent{0mm}
\rhead{\textbf{A. P{}. M. E. P{}.}}
\lhead{\small Corrigé du brevet des collèges}
\lfoot{\small{Amérique du Nord}}
\rfoot{\small{7 juin 2013}}
\renewcommand \footrulewidth{.2pt}
\pagestyle{fancy}
\thispagestyle{empty}
\begin{center}
\textbf{DurŽée : 2 heures}

\vspace{0,25cm}

{\Large \textbf{\decofourleft~Corrigé du brevet des collèges Amérique du Nord 7 juin 2013~\decofourright}}

\vspace{0,25cm}

L'utilisation d'une calculatrice est autorisŽée.

\end{center}
 
\vspace{0,25cm}

\textbf{\textsc{Exercice 1} \hfill 4 points}

\medskip

%\emph{Pour chacune des quatre questions suivantes, plusieurs propositions de réponse sont faites. Une seule des propositions est exacte. Aucune justification n'est attendue. Une bonne réponse rapporte $1$ point. Une mauvaise réponse ou une absence de réponse rapporte $0$ point. Reporter sur votre copie le numéro de la question et donner la bonne réponse.}
%
%\medskip
 
\begin{enumerate}
\item %L'arbre ci-dessous est un arbre de probabilité.

%\begin{center}
%\psset{nrot=:U}
%\pstree[treemode=R]{\Tdot}
%{\Tdot~[tnpos=r]{}\naput{$\frac{1}{9}$}
%\Tdot~[tnpos=r]{}\nbput{\psset{unit=1cm}\begin{pspicture}(-0.25,0)(0.25,0.25)\pscurve*(-0.25,0.25)(-0.15,0.45)(0,0.5)(0.25,0.5)(0.25,0.25)(0.1,0.05)(0,0.1)(-0.15,0.15)(-0.25,0.25)\end{pspicture}}
% \Tdot~[tnpos=r]{}\nbput{$\frac{1}{3}$}
% } 
% \end{center}

%La probabilité manquante sous la tache est: 

%\medskip
%\begin{tabularx}{\linewidth}{*{3}{X}} 
%\textbf{a.~~} $\dfrac{7}{9}$ &\textbf{b.~~} $\dfrac{5}{12}$ &\textbf{c.~~} $\dfrac{5}{9}$
%\end{tabularx}
%\medskip
La somme des probabilités est égale à 1  : la probabilité manquante est donc 

$1 - \left(\dfrac{1}{9} + \dfrac{1}{3} \right) = 1 - \left(\dfrac{1}{9} + \dfrac{3}{9} \right) = 1 - \dfrac{4}{9} = \dfrac{5}{9}$. Réponse \textbf{c.} 
\item %Dans une salle, il y a des tables à 3 pieds et à 4 pieds. Léa compte avec les yeux bandés 169 pieds. Son frère lui indique qu'il y a 34 tables à 4 pieds. Sans enlever son bandeau, elle parvient à donner le nombre de tables à 3 pieds qui est de :
 
%\medskip
%\begin{tabularx}{\linewidth}{*{3}{X}} 
%\textbf{a.~~} 135&\textbf{b.~~} 11&\textbf{c.~~} 166 
%\end{tabularx}
S'il y a $t$ tables à trois pieds et $34$ tables à quatre pieds, on a :

$3t + 4 \times 34 = 169$ soit $3t + 136 = 169$ ou encore $3t = 33$ et enfin $t = 11$. Réponse \textbf{b.} 
\medskip
\item %90\,\% du volume d'un iceberg est situé sous la surface de l'eau.
 
%La hauteur totale d'un iceberg dont la partie visible est 35~m est d'environ: 

%\medskip
%\begin{tabularx}{\linewidth}{*{3}{X}}
%\textbf{a.~~}  350 m&\textbf{b.~~} \np{3500} m&\textbf{c.~~} 31,5 m
%\end{tabularx}
La partie visible représente 10\,\%, soit 35~m, donc l'iceberg mesure 350~m. Réponse \textbf{a.}
\medskip 
\item %\psset{unit=0.6cm}\begin{pspicture}(3.1,2.3)
%\psline(1.2,0.6)(0,0.6)(0,2.1)(1.2,2.1)
%\psline(2.4,2.1)(3.1,2.1)(3.1,0.6)(2.4,0.6)
%\psarc(1.8,2.1){0.6}{-180}{0}
%\psarc(1.8,0.6){0.6}{-180}{0}
%\end{pspicture} a le même périmètre que: 

%\medskip
%\begin{tabularx}{\linewidth}{*{3}{X}} 
%\textbf{a.~~}  &\textbf{b.~~}  &\textbf{c.~~}\\
%\psset{unit=0.6cm}\begin{pspicture}(3.2,2.5) 
%\psframe(0,0)(3.1,2.1)\end{pspicture}&
%\psset{unit=0.6cm}\begin{pspicture}(3.1,2.3)\psline(1.2,0.6)(0,0.6)(0,2.1)(1.2,2.1)
%\psline(2.4,2.1)(3.1,2.1)(3.1,0.6)(2.4,0.6)
%\psarc(1.8,2.1){0.6}{0}{180}
%\psarc(1.8,0.6){0.6}{180}{0}
%\end{pspicture}&\psset{unit=0.6cm}\begin{pspicture}(3.1,2.3)
%\psline(1.2,0.6)(0,0.6)(0,2.1)(3.1,2.1)(3.1,0.6)(2.4,0.6)
%\psarc(1.8,0.6){0.6}{180}{0}
%\end{pspicture} 
%\end{tabularx}
Réponse \textbf{b.}
\medskip 
\end{enumerate}

\bigskip

\textbf{\textsc{Exercice 2} \hfill 4 points}

\medskip 

%Arthur vide sa tirelire et constate qu'il possède 21 billets.
% 
%Il a des billets de 5~\euro{} et des billets de 10~\euro{} pour une somme totale de 125~\euro.
%
%\medskip
% 
%Combien de billets de chaque sorte possède-t-il ? 
S'il a $c$ billets de cinq \euro, il a $21 - c$ billets de 10~\euro.

Il a donc :
$5c + 10(21 - c) = 125$~(\euro) soit $5c + 210 - 10c = 125$ et $5c = 85$.

Finalement $c = 17$ ; Arthur a $21 - 17 = 4$ billets de 10~\euro{} et 17 billets de 5~\euro.

%\medskip
%
%\textbf{Si le travail n'est pas terminé, laisse tout de même une trace de la recherche. Elle sera prise en compte dans l'évaluation.}

\bigskip

\textbf{\textsc{Exercice 3} \hfill 6 points}

\medskip

%Caroline souhaite s'équiper pour faire du roller.
% 
%Elle a le choix entre une paire de rollers gris à 87~\euro{} ? et une paire de rollers noirs à 99~\euro.
% 
%Elle doit aussi acheter un casque et hésite entre trois modèles qui coûtent respectivement 45~\euro, 22~\euro{} et 29~\euro.
%
%\medskip
 
\begin{enumerate}
\item ~
%Si elle choisit son équipement (un casque et une paire de rollers) au hasard, quelle est la probabilité pour que l'ensemble lui coûte moins de 130~\euro{}?
\begin{center}
\pstree[treemode=R,nodesepA=0pt,nodesepB=2.5pt]{\TR{}}
{\pstree{\TR{87}}
	{\TR{45 $\to$ 132}
	\TR{22   $\to$ 109}
	\TR{29 $\to$ 116}
	}
\pstree{\TR{99}}
	{\TR{45  $\to$ 144}
	\TR{22  $\to$ 121}
	\TR{29  $\to$ 128}
	}
}
\end{center}
Sur les six possibilités quatre reviennent à moins de 130~\euro. La probabilité est donc égale à : $\dfrac{4}{6} = \dfrac{2}{3}$. 
\item %Elle s'aperçoit qu'en achetant la paire de rollers noirs et le casque à 45~\euro, elle bénéficie d'une réduction de 20\,\% sur l'ensemble.
Prix avant réduction : $99 + 45 = 144$~\euro 
	\begin{enumerate}
		\item %Calculer le prix en euros et centimes de cet ensemble après réduction.
Avoir 20\,\% de réduction c'est payer 80\,\% du prix initial soit : 

$0,80 \times 144 = 115,20$~\euro. 
		\item %Cela modifie-t-il la probabilité obtenue à la question 1 ? Justifier la réponse.
Avec cette réduction le prix passe en dessous de 130~\euro ; la probabilité est donc maintenant égale à  $\dfrac{5}{6}$.
	\end{enumerate}
\end{enumerate}
 
\bigskip

\textbf{\textsc{Exercice 4} \hfill 5 points}

\medskip

Flavien veut répartir la totalité de 760~dragées au chocolat et \np{1045}~dragées aux amandes dans des sachets dans des sachets ayant la même répartition de dragées au chocolat et aux amandes.

\medskip
 
\begin{enumerate}
\item %Peut-il faire 76 sachets ? Justifier la réponse.
On a $760 = 76 \times 10$ mais  $\np{1045}$ impair ne peut être multiple de 76 qui est pair. On ne peut donc répartir ces dragées dans 76 sachets.
\item 
	\begin{enumerate}
		\item %Quel nombre maximal de sachets peut-il réaliser ?
		On cherche avec l'algorithme d'Euclide le PGCD à 760 et \np{1045} :
		
$\np{1045} = 760 \times 1 + 285$ ;

$760 = 285 \times 2 + 190$ ;

$285 = 190 \times 1 + 95$ ;

$190 = 95 \times 2 + 0$.

On a donc PGCD(760~;~\np{1045}) = 95$.

On peut faire au maximum 95 sachets. 
		\item %Combien de dragées de chaque sorte y aura-t-il dans chaque sachet ?
On  a $760 = 95 \times 8$ et $\np{1045} = 95 \times 11$.

Il y a dans chacun des 95 sachets, 8 dragées au chocolat et 11 dragées aux amandes.
	\end{enumerate}
\end{enumerate}
 
\vspace{0,5cm}

\textbf{\textsc{Exercice 5} \hfill 4 points}

\bigskip
 
%Tom doit calculer $3,5^2$.
% 
%\og Pas la peine de prendre la calculatrice \fg, lui dit Julie, tu n'as qu'à effectuer le produit de $3$ par $4$ et rajouter $0,25$. 
%
%\medskip

\begin{enumerate}
\item %Effectuer le calcul proposé par Julie et vérifier que le résultat obtenu est bien le carré de $3,5$.
$3 \times 4  + 0,25 = 12 + 0,25 = 12,25$.

Or $3,5^2 = \left(\dfrac{7}{2} \right)^2 = \dfrac{7^2}{2^2} = \dfrac{49}{4} = \dfrac{24,5}{2} = 12,25$. Le calcul est exact. 
\item %Proposer une façon simple de calculer $7,5^2$ et donner le résultat.
Multiplier 7 par 8 et ajouter 0,25 au produit.

$7 \times 8 + 0,25 = 56,25$.

$7,5^2 = \left(\dfrac{15}{2} \right)^2 = \dfrac{15^2}{2^2} = \dfrac{225}{4} = \dfrac{112,5}{2} = 56,25$. Exact ! 
\item %Julie propose la conjecture suivante : 	$(n + 0,5)^2 = n(n + 1) + 0,25$ 

%$n$ est un nombre entier positif. 

%Prouver que la conjecture de Julie est vraie (quel que soit le nombre $n$)
Quel que soit le naturel $n$ : $(n + 0,5)^2 = n^2 + 0,5^2 + 2 \times n \times 0,5 = n^2 + n + 0,25 = n(n + 1) + 0,25$.

La  conjecture de Julie est vraie.
\end{enumerate} 

\bigskip

\textbf{\textsc{Exercice 6} \hfill 4 points}

\medskip
 
%On dispose d'un carré de métal de $40$ cm de côté. Pour fabriquer une boîte parallélépipèdique, on enlève à chaque coin un carré de côté $x$ et on relève les bords par pliage.
%
%\medskip
 
\begin{enumerate}
\item %Quelles sont les valeurs possibles de x ? 
On enlève en tout $2x$ de 40, donc $0 \leqslant x \leqslant 20$.
\item %On donne $x = 5$ cm. Calculez le volume de la boîte.

On a donc un pavé de fond carré de côtés mesurant $40 - 2 \times 5 = 30$ et de hauteur 5.

Le volume du pavé est donc égal à $30 \times 30 \times 5 = 900 \times 5 = \np{4500$~cm$^3$. 
\item %Le graphique suivant donne le volume de la boîte en fonction de la longueur $x$.

%\emph{On répondra aux questions à l'aide du graphique.} 
	\begin{enumerate}
		\item %Pour quelle valeur de $x$, le volume de la boîte est-il maximum ?
Le maximum semble atteint pour $x = 6,5$. 
		\item %On souhaite que le volume de la boîte soit \np{2000}~cm$^3$. 
		
%Quelles sont les valeurs possibles de $x$ ?
La droite d'équation $y = \np{2000}$ coupe la courbe aux points d'abscisse 1,5 et 14.
	\end{enumerate} 
\end{enumerate}

\begin{center}
%\psset{unit=0.8cm}
%\begin{pspicture}(15,6)
%\pspolygon[fillstyle=solid,fillcolor=lightgray](2,0)(5.2,0)(5.2,0.9)(6,0.9)(6,4.1)(5.2,4.1)(5.2,4.9)(2,4.9)(2,4.1)(1.2,4.1)(1.2,0.9)(2,0.9)
%\psline[arrowsize=2pt 3]{<->}(1.2,5.2)(6,5.2)
%\psline[arrowsize=2pt 3]{<->}(0.7,4.1)(0.7,4.9)
%\psline[linestyle=dashed](1.2,4)(1.2,5.7)
%\psline[linestyle=dashed](0.3,4.9)(2,4.9)
%\psline[linestyle=dashed](6,4)(6,5.4)
%\psline[linestyle=dashed](1.2,4.1)(0.3,4.1)
%\pspolygon[fillstyle=solid,fillcolor=gray](9.3,2)(11.4,3.4)(13.3,2.85)(12.05,1.85)(12.05,0.8)(10.4,1.6)
%\pspolygon[fillstyle=solid,fillcolor=lightgray](7.7,2.5)(9.3,2)(10.4,1.6)(12.05,0.8)(9.2,1.7)%petit bord
%\pspolygon[fillstyle=solid,fillcolor=lightgray](12.05,0.8)(12.05,1.95)(14.45,3.65)(14.45,2.55)%bord vertical droit
%\pspolygon[fillstyle=solid,fillcolor=lightgray](11.4,3.4)(11.4,4.65)(14.45,3.65)(13.3,2.85)%bord vertical fond
%\pspolygon[fillstyle=solid,fillcolor=lightgray](8.7,2.2)(8.2,2.75)(10.5,4.5)(11.4,3.4)(9.3,2)%bord penché gauche
%\uput[u](3.6,5.3){40}
%\uput[l](0.7,4.5){$x$}
%\end{pspicture}

\vspace{0,5cm}
\psset{xunit=0.5cm,yunit=0.001cm}
\begin{pspicture}(-1,-500)(21,5500)
\multido{\n=0.0+0.5}{43}{\psline[linewidth=0.2pt,linecolor=orange](\n,0)(\n,5500)}
\multido{\n=0+250}{23}{\psline[linewidth=0.2pt,linecolor=orange](0,\n)(21,\n)}
\psaxes[linewidth=1.5pt,Dy=6000,labelFontSize=\scriptstyle]{->}(0,0)(21,5500)
\multido{\n=0+500}{11}{\uput[l](0,\n){\footnotesize \np{\n}}}
\uput[d](21,-200){$x$}
\uput[r](0,5500){volume de la boîte}
\psplot[plotpoints=8000,linewidth=1.25pt,linecolor=blue]{0}{20}{40 x 2 mul  sub dup mul x mul}
\uput[dl](0,0){O}
\end{pspicture}
\end{center} 

\bigskip

\textbf{\textsc{Exercice 7} \hfill 5 points}

\medskip 

%\parbox{0.6\linewidth}{Le Pentagone est un bâtiment hébergeant le ministère de la défense des Etats-Unis.
% 
%Il a la forme d'un pentagone régulier inscrit dans un cercle de rayon OA= 238 m.
% 
%Il est représenté par le schéma ci-contre.}\hfill 
%\parbox{0.38\linewidth}{\psset{unit=1.25cm}\begin{pspicture}(-1.5,-1.5)(1.5,1.5)
%\pspolygon(1.2;20)(1.2;92)(1.2;164)(1.2;236)(1.2;308)
%\psline(1.2;92)(0;0)(1.2;164)
%\psline(0;0)(0.97;128)
%\uput[u](1.2;92){\footnotesize A} \uput[ul](1.2;164){\footnotesize B} \uput[dl](1.2;236){\footnotesize C} 
%\uput[dr](1.2;308){\footnotesize D} \uput[ur](1.2;20){\footnotesize E} \uput[dr](0,0){\footnotesize O} 
%\uput[ul](0.97;128){\footnotesize M}
%\rput{-52}(0.97;128){\psframe(0.2,0.2)}  
%\end{pspicture}}

\medskip

\begin{enumerate}
\item %Calculer la mesure de l'angle $\widehat{\text{AOB}}$.
Puisque le polygone est régulier les cinq angles au centre ont la même mesure soit $\dfrac{360}{5} = 72$\degres. 
\item %La hauteur issue de O dans le triangle AOB coupe le côté [AB] au point M. 
	\begin{enumerate}
		\item %Justifier que (OM) est aussi la bissectrice de $\widehat{\text{AOB}}$ et la médiatrice de [AB].
OA = OB montre que le triangle OAB est isocèle  ; la hauteur [OM] est aussi la médiatrice de [AB] (le théorème de Pythagore appliqué aux triangles OAM et OBM montre que MA = MB, donc M et O sont équidistants de A et de B) et la bissectrice de l'angle $\widehat{\text{BOA}}$. 
Donc $\widehat{\text{AOM}} = 36$\degres.
		\item %Prouver que [AM] mesure environ $140$~m.
		Dans le triangle OAM rectangle en M, on a $\sin \widehat{\text{AOB}} = \dfrac{\text{AM}}{\text{AO}}$ ; donc 
		
AM $ =  {\text{AO}} \times \sin \widehat{\text{AOB}} = 238 \sin 36 \approx 139,89$, soit au mètre près 140~m..
		\item %En déduire une valeur approchée du périmètre du Pentagone.
		Chaque côté mesure donc $2 \times 140 = 180$ et le périmètre est donc égal à $5 \times 280 = \np{1400}$~m.
	\end{enumerate}
\end{enumerate}
 
\bigskip

\textbf{\textsc{Exercice 8} \hfill 4 points}

\medskip  

%\parbox{0.5\linewidth}{Les longueurs sont données en centimètres. 
%
%ABCD est un trapèze.}\hfill
%\parbox{0.48\linewidth}{\psset{unit=0.8cm}
%\begin{pspicture}(7,3)
%\pspolygon(0,0)(1,3)(4,3)(7,0)
%\psline[linestyle=dashed](0,0)(0,3)(1,3)
%\psline[linestyle=dashed](4,3)(7,3)(7,0)
%\psframe(0,3)(0.2,2.8)
%\psframe(7,3)(6.8,2.8)
%\rput(2.5,3){o}\rput(5.5,3){o}\rput(7,1.5){o}
%\uput[r](7,1.5){3}\uput[u](0.5,3){1}\uput[d](3.5,0){7}
%\uput[ur](1,3){A} \uput[ur](4,3){B} \uput[dr](7,0){C} \uput[dl](0,0){D} 
%\end{pspicture}} 
%
%\medskip

\begin{enumerate}
\item 
	\begin{enumerate}
		\item %Donner une méthode permettant de calculer l'aire du trapèze ABCD.
\emph{Méthode $1$} : on part de l'aire du rectangle à laquelle on retire l'aire des deux triangles rectangles :

$7 \times 3 - \left(\dfrac{1}{2} \times 3 \times 1 \right)  - \left(\dfrac{1}{2} \times 3 \times  3\right) = 21 -  \dfrac{3}{2} - \dfrac{9}{2} = 21 - 6 = 15$~cm$^2$.

\emph{Méthode $1$} : on utilise la formule de l'aire du trapèze : 

$\dfrac{(B + b) \times h}{2} = \dfrac{(7 + 3)\times 3}{2} = 15$~cm$^2$.
		\item %Calculer l'aire de ABCD.
		Voir ci-dessus.
	\end{enumerate} 
\item %\textbf{Dans cette question, si le travail n'est pas terminé, laisser tout de même une trace de la recherche. Elle sera prise en compte dans l'évaluation.}
 
%L'aire d'un trapèze $A$ est donnée par l'une des formules suivantes. Retrouver la formule juste en expliquant votre choix.
C'est la deuxième expression qui est correcte.

Il suffit de racer une diagonale du trapèze pour retrouver cette formule  : l'aire du trapèze est la somme des aires de deux triangles : $\dfrac{Bh}{2} + \dfrac{bh}{2} = \dfrac{(b + B)h}{2}$.

\begin{center}
\psset{unit=0.8cm}
\begin{pspicture}(0,-0.2)(7,3.2)
\pspolygon(0,0)(1,3)(4,3)(7,0)
\psline[linestyle=dotted,linecolor=blue](0,0)(4,3)
\psline[linestyle=dashed](4,3)(7,3)(7,0)
\psframe(7,3)(6.8,2.8)
\uput[u](2.5,3){$b$}
\uput[d](3.5,0){$B$}
\uput[r](7,1.5){$h$} 
\end{pspicture}
\end{center}
\medskip

%\begin{tabularx}{\linewidth}{*{3}{X}}
%$A = \dfrac{(b . B)h}{2}$& 
%$A = \dfrac{(b + B)h}{2}$& 
%$A = 2(b + B)h$
%\end{tabularx} 
\end{enumerate}
\end{document}